\documentclass[onecolumn, amsmath, amsfonts, amssymb]{aastex62}
\usepackage{mathtools}
\usepackage{natbib}
\usepackage{bm}
\newcommand{\vdag}{(v)^\dagger}
\newcommand\aastex{AAS\TeX}
\newcommand\latex{La\TeX}


\newcommand{\Div}[1]{\ensuremath{\nabla\cdot\left( #1\right)}}
\newcommand{\DivU}{\ensuremath{\nabla\cdot\bm{u}}}
\newcommand{\angles}[1]{\ensuremath{\left\langle #1 \right\rangle}}
\newcommand{\KS}[1]{\ensuremath{\text{KS}(#1)}}
\newcommand{\KSstat}[1]{\ensuremath{\overline{\text{KS}(#1)}}}
\newcommand{\grad}{\ensuremath{\nabla}}
\newcommand{\RB}{Rayleigh-B\'{e}nard }
\newcommand{\stressT}{\ensuremath{\bm{\bar{\bar{\Pi}}}}}
\newcommand{\lilstressT}{\ensuremath{\bm{\bar{\bar{\sigma}}}}}
\newcommand{\nrho}{\ensuremath{n_{\rho}}}
\newcommand{\approptoinn}[2]{\mathrel{\vcenter{
	\offinterlineskip\halign{\hfil$##$\cr
	#1\propto\cr\noalign{\kern2pt}#1\sim\cr\noalign{\kern-2pt}}}}}

\newcommand{\appropto}{\mathpalette\approptoinn\relax}
\newcommand{\pro}{\ensuremath{\text{Ro}_{\text{p}}}}
\newcommand{\con}{\ensuremath{\text{Ro}_{\text{c}}}}

\usepackage{color}
\newcommand{\gv}[1]{{\color{blue} #1}}

%% Tells LaTeX to search for image files in the 
%% current directory as well as in the figures/ folder.
\graphicspath{{./}{figs/}}

\begin{document}
\section{Hydrodynamic Impulse}
The theory that we're going to use to describe thermals relies on hydrodynamic impulse,
and since we happen to be looking at \emph{stratified} domains, we're lucky that
\citet{shivamoggi2010} went through the work of deriving impulse in anelastic atmospheres.
He finds that for an anelastic fluid (e.g., one where $\Div{\rho_0\bm{u}} = 0$ for 
a constant $\rho_0(z)$), the impulse is
\begin{equation}
\bm{I} = \frac{1}{2}\int_{\mathcal{V}} \bm{r}\times(\nabla\times(\rho_0\bm{u}))d\bm{r},
\label{eqn:impulse}
\end{equation}
where $\bm{r} \equiv x\hat{x} + y\hat{y} + z\hat{z}$ is the position vector and $\mathcal{V}$ is a closed volume.
Basically our theory will consist of two parts:
\begin{enumerate}
\item An expectation for the time-derivative of the impulse of the thermal based on its
density excess (and/or other thermodynamic properties)
\item The time-derivative of an approximate expression for the thermal's impulse when it is
in its evolved vortex-ring state.
\end{enumerate}
Our expectation is that these two things will be approximately equal, and will tell us something
about the size of the thermal as it descends and thus its entrainment.

\subsection{Volumetric \& Surface terms of the impulse}
For the first part of our theory we'll need to find the time-derivative of the impulse. Before we do that, 
we need to break apart the impulse expression into surface terms and volumetric terms (as Daniel did in an email
on 9/4/2018). First, the component of the impulse in the $i$th direction for $i$ in $(x,y,z)$ is
$$
[\bm{r}\times(\grad\times(\rho_0\bm{u}))]_i = 
\grad_i(\rho_0\bm{u}) \cdot \bm{r}- (\bm{r}\cdot\grad)(\rho_0 u)_i = 
r_j\frac{\partial}{\partial x_i}(\rho_0 u)_j - r_j \frac{\partial}{\partial x_j}(\rho_0 u)_i.
$$
By definition, each of the RHS terms can be further expanded as
$$
r_j\frac{\partial}{\partial x_i}(\rho_0 u)_j =
\frac{\partial}{\partial x_i}(r_j(\rho_0 u)_j) - \delta_{ij}(\rho_0 u)_j,\qquad\text{and}
\qquad
r_j \frac{\partial}{\partial x_j}(\rho_0 u)_i =
\frac{\partial}{\partial x_j}(r_j(\rho_0 u)_i) - 3(\rho_0 u)_i.
$$
Plugging these expanded expressions in, we find that
$$
[\bm{r}\times(\grad\times(\rho_0\bm{u}))]_i = 
\frac{\partial}{\partial x_i}(r_j(\rho_0 u)_j) -
\frac{\partial}{\partial x_j}(r_j(\rho_0 u)_i) +
2(\rho_0 u)_i.
$$
Plugging back in to the full impulse defn in Eqn.~\ref{eqn:impulse}, we find
\begin{equation}
\bm{I} = \int_{\mathcal{V}} (\rho_0 \bm{u}) d\bm{r} + 
\frac{1}{2}\int_{\mathcal{V}} [\grad (\bm{r}\cdot(\rho_0 \bm{u})) - \grad \cdot(\bm{r}(\rho_0\bm{u}))] d\bm{r}.
\end{equation}
As this integral is being done over a closed volume, the last integral can be expressed in terms of surface terms,
\begin{equation}
\bm{I} = \int_{\mathcal{V}} (\rho_0 \bm{u}) d\bm{r} + 
\frac{1}{2}\int_{\mathcal{S}} d\bm{S}(\bm{r}\cdot(\rho_0 \bm{u})) - d\bm{S} \cdot(\bm{r}(\rho_0\bm{u})).
\end{equation}



\bibliography{biblio.bib}
\end{document}
