\documentclass[onecolumn, amsmath, amsfonts, amssymb]{aastex62}
\usepackage{mathtools}
\usepackage{natbib}
\usepackage{bm}
\newcommand{\vdag}{(v)^\dagger}
\newcommand\aastex{AAS\TeX}
\newcommand\latex{La\TeX}


\newcommand{\Div}[1]{\ensuremath{\nabla\cdot\left( #1\right)}}
\newcommand{\DivU}{\ensuremath{\nabla\cdot\bm{u}}}
\newcommand{\angles}[1]{\ensuremath{\left\langle #1 \right\rangle}}
\newcommand{\KS}[1]{\ensuremath{\text{KS}(#1)}}
\newcommand{\KSstat}[1]{\ensuremath{\overline{\text{KS}(#1)}}}
\newcommand{\grad}{\ensuremath{\nabla}}
\newcommand{\RB}{Rayleigh-B\'{e}nard }
\newcommand{\stressT}{\ensuremath{\bm{\bar{\bar{\Pi}}}}}
\newcommand{\lilstressT}{\ensuremath{\bm{\bar{\bar{\sigma}}}}}
\newcommand{\nrho}{\ensuremath{n_{\rho}}}
\newcommand{\approptoinn}[2]{\mathrel{\vcenter{
	\offinterlineskip\halign{\hfil$##$\cr
	#1\propto\cr\noalign{\kern2pt}#1\sim\cr\noalign{\kern-2pt}}}}}

\newcommand{\appropto}{\mathpalette\approptoinn\relax}
\newcommand{\pro}{\ensuremath{\text{Ro}_{\text{p}}}}
\newcommand{\con}{\ensuremath{\text{Ro}_{\text{c}}}}

\usepackage{color}
\newcommand{\gv}[1]{{\color{blue} #1}}

%% Tells LaTeX to search for image files in the 
%% current directory as well as in the figures/ folder.
\graphicspath{{./}{figs/}}

\begin{document}
\section{Hydrodynamic Impulse}
The theory that we're going to use to describe thermals relies on hydrodynamic impulse,
and since we happen to be looking at \emph{stratified} domains, we're lucky that
\citet{shivamoggi2010} went through the work of deriving impulse in anelastic atmospheres.
He finds that for an anelastic fluid (e.g., one where $\Div{\rho_0\bm{u}} = 0$ for 
a constant $\rho_0(z)$), the impulse is
\begin{equation}
\bm{I} = \frac{1}{2}\int_{\mathcal{V}} \bm{r}\times(\nabla\times(\rho_0\bm{u}))d\bm{r},
\label{eqn:impulse}
\end{equation}
where $\bm{r} \equiv x\hat{x} + y\hat{y} + z\hat{z}$ is the position vector and $\mathcal{V}$ is a closed volume.
Basically our theory will consist of two parts:
\begin{enumerate}
\item An expectation for the time-derivative of the impulse of the thermal based on its
density excess (and/or other thermodynamic properties)
\item The time-derivative of an approximate expression for the thermal's impulse when it is
in its evolved vortex-ring state.
\end{enumerate}
Our expectation is that these two things will be approximately equal, and will tell us something
about the size of the thermal as it descends and thus its entrainment.

\subsection{Volumetric \& Surface terms of the impulse}
For the first part of our theory we'll need to find the time-derivative of the impulse. Before we do that, 
we need to break apart the impulse expression into surface terms and volumetric terms (as Daniel did in an email
on 9/4/2018). First, the component of the impulse in the $i$th direction for $i$ in $(x,y,z)$ is
$$
[\bm{r}\times(\grad\times(\rho_0\bm{u}))]_i = 
\grad_i(\rho_0\bm{u}) \cdot \bm{r}- (\bm{r}\cdot\grad)(\rho_0 u)_i = 
r_j\frac{\partial}{\partial x_i}(\rho_0 u)_j - r_j \frac{\partial}{\partial x_j}(\rho_0 u)_i.
$$
By definition, each of the RHS terms can be further expanded as
$$
r_j\frac{\partial}{\partial x_i}(\rho_0 u)_j =
\frac{\partial}{\partial x_i}(r_j(\rho_0 u)_j) - \delta_{ij}(\rho_0 u)_j,\qquad\text{and}
\qquad
r_j \frac{\partial}{\partial x_j}(\rho_0 u)_i =
\frac{\partial}{\partial x_j}(r_j(\rho_0 u)_i) - 3(\rho_0 u)_i.
$$
Plugging these expanded expressions in, we find that
$$
[\bm{r}\times(\grad\times(\rho_0\bm{u}))]_i = 
\frac{\partial}{\partial x_i}(r_j(\rho_0 u)_j) -
\frac{\partial}{\partial x_j}(r_j(\rho_0 u)_i) +
2(\rho_0 u)_i.
$$
Plugging back in to the full impulse defn in Eqn.~\ref{eqn:impulse}, we find
\begin{equation}
\bm{I} = \int_{\mathcal{V}} (\rho_0 \bm{u}) d\bm{r} + 
\frac{1}{2}\int_{\mathcal{V}} [\grad (\bm{r}\cdot(\rho_0 \bm{u})) - \grad \cdot(\bm{r}(\rho_0\bm{u}))] d\bm{r}.
\end{equation}
As this integral is being done over a closed volume, the last integral can be expressed in terms of surface terms,
\begin{equation}
\bm{I} = \int_{\mathcal{V}} (\rho_0 \bm{u}) d\bm{r} + 
\frac{1}{2}\int_{\mathcal{S}} d\bm{S}(\bm{r}\cdot(\rho_0 \bm{u})) - d\bm{S} \cdot(\bm{r}(\rho_0\bm{u})).
\label{eqn:surface_impulse}
\end{equation}

\subsection{Time derivative of the impulse}
The time derivative of the impulse is \citep{shivamoggi2010},
\begin{equation}
\frac{\partial \bm{I}}{\partial t} = \int_{\mathcal{V}}\frac{\partial}{\partial t}(\rho_0 \bm{u}) d\bm{r}.
\end{equation}
I'm not totally convinced this is right. There's not much explanation of this in the paper. In an unbounded
fluid (which I think is what he's using this expression for), it makes sense that the surface terms
in Eqn.~\ref{eqn:surface_impulse} drop out. I'm not entirely sure why $d/dt \rightarrow \partial/\partial t$
inside of the derivative. Given a fairly standard combination of an ideal 
momentum equation and continuity equation,
$$
\frac{\partial}{\partial t}{\rho_0 \bm{u}} + \grad\cdot(\rho_0\bm{u}\bm{u}) = -\grad P - (\rho_0 + \rho_1) g \hat{z}.
$$
Now technically this whole $\rho_1$ term doesn't jive with the anelastic approximation (where $\rho$ can't change),
but I'm hesitant to put the equations into a $\varpi$ and $S_1$ formulation, because I don't think they accurately describe
the thermal's evolution. Subtracting out hydrostatic balance, we have something like
$$
\frac{\partial}{\partial t}{\rho_0 \bm{u}} = - \grad\cdot(\rho_0\bm{u}\bm{u}) - \grad P_1 - \rho_1 g \hat{z}.
$$
and thus the change in impulse over time is
\begin{equation}
\frac{\partial \bm{I}}{\partial t} = 
\int_{\mathcal{V}}\left(
- \grad\cdot(\rho_0\bm{u}\bm{u}) - \grad P_1 - \rho_1 g \hat{z}
\right)d\bm{r}
= -\int_{\mathcal{V}}\rho_1 g\hat{z} d\bm{r} -
\int_{\mathcal{S}} [d\bm{S}\cdot(\rho_0\bm{u}\bm{u}) + d\bm{S}P_1].
\end{equation}
Generally (in \citet{shivamoggi2010} and I think other thermal work), the
surface terms here are said to be zero (at least, under appropriate boundary
conditions), and so we're left with
\begin{equation}
\boxed{
\frac{d\bm{I}}{dt} = -gM_1 \hat{z}, \qquad M_1 \equiv \int_{\mathcal{V}}\rho_1 dV
}.
\end{equation}
Or, your change in impulse over time is directly proportional to the excess mass
in the thermal.

\subsection{Approximate expression for the impulse}
Going back to Eqn.~\ref{eqn:impulse}, we can see that the impulse contains two expressions:
\begin{equation}
\bm{I} = \frac{1}{2}\int_{\mathcal{V}}\bm{r}\times(\grad\times(\rho_0\bm{u}))d\bm{r} 
= \frac{1}{2} \int_{\mathcal{V}} \left(\rho_0 \bm{r}\times \bm{\omega} + \bm{r} \times(\grad\rho_0 \times \bm{u}) \right) d\bm{r},
\end{equation}
where $\bm{\omega} = \grad\times\bm{u}$ is the vorticity.  The first term is identical to the term that shows up
in boussinesq flows, and is basically \citep{saffman1970}
$$
\frac{1}{2} \int_{\mathcal{V}} \rho_0 \bm{r}\times \bm{\omega} = \rho_0 c R^2 \Gamma \hat{z},
$$
where $R$ is the radius of the vortex ring, $\Gamma$ is its circulation, and $c$ is some constant describing its
aspect ratio. The second term here is
$$
\bm{r}\times(\grad\rho_0 \times\bm{u}) = (\bm{r}\cdot\bm{u})\grad\rho_0 - (\bm{r}\cdot\grad\rho_0)\bm{u}).
$$
Assuming that $\grad \rho_0 = \partial_z \rho_0 \hat{z}$, this becomes
$$
\bm{r}\times(\grad\rho_0 \times\bm{u}) = (-uz\hat{x} - vz\hat{y} + (xu + yv)\hat{z}).
$$
We don't care about the $x$ and $y$ terms here, because the main impulse of the thermal is vertical. We're essentially left with
$$
I_z = \rho_0 c R^2 \Gamma + \frac{1}{2}\int_{\mathcal{V}} u_{\perp} r_{\perp} \frac{\partial \rho_0}{\partial z} \bm{r}.
$$
If the thermal is small, and $\partial \rho_0 /\partial z$ is constant across the thermal, the last term drops out
($u_{\perp} r_{\perp}$ at the bottom of the thermal cancels out with $u_{\perp} r_{\perp}$ at the top of the thermal).
If the thermal is large and the stratification of the background varies across the thermal's size, then we need to worry
about that second term.

\bibliography{biblio.bib}
\end{document}
