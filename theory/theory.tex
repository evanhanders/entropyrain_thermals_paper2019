\documentclass[onecolumn, amsmath, amsfonts, amssymb]{aastex62}
\usepackage{mathtools}
\usepackage{natbib}
\usepackage{bm}
\newcommand{\vdag}{(v)^\dagger}
\newcommand\aastex{AAS\TeX}
\newcommand\latex{La\TeX}


\newcommand{\Div}[1]{\ensuremath{\nabla\cdot\left( #1\right)}}
\newcommand{\DivU}{\ensuremath{\nabla\cdot\bm{u}}}
\newcommand{\angles}[1]{\ensuremath{\left\langle #1 \right\rangle}}
\newcommand{\KS}[1]{\ensuremath{\text{KS}(#1)}}
\newcommand{\KSstat}[1]{\ensuremath{\overline{\text{KS}(#1)}}}
\newcommand{\grad}{\ensuremath{\nabla}}
\newcommand{\RB}{Rayleigh-B\'{e}nard }
\newcommand{\stressT}{\ensuremath{\bm{\bar{\bar{\Pi}}}}}
\newcommand{\lilstressT}{\ensuremath{\bm{\bar{\bar{\sigma}}}}}
\newcommand{\nrho}{\ensuremath{n_{\rho}}}
\newcommand{\approptoinn}[2]{\mathrel{\vcenter{
	\offinterlineskip\halign{\hfil$##$\cr
	#1\propto\cr\noalign{\kern2pt}#1\sim\cr\noalign{\kern-2pt}}}}}

\newcommand{\appropto}{\mathpalette\approptoinn\relax}
\newcommand{\pro}{\ensuremath{\text{Ro}_{\text{p}}}}
\newcommand{\con}{\ensuremath{\text{Ro}_{\text{c}}}}

\usepackage{color}
\newcommand{\gv}[1]{{\color{blue} #1}}

%% Tells LaTeX to search for image files in the 
%% current directory as well as in the figures/ folder.
\graphicspath{{./}{figs/}}

\begin{document}
\section{Hydrodynamic Impulse}
The theory that we're going to use to describe thermals relies on hydrodynamic impulse,
and since we happen to be looking at \emph{stratified} domains, we're lucky that
\citet{shivamoggi2010} went through the work of deriving impulse in anelastic atmospheres.
He finds that for an anelastic fluid (e.g., one where $\Div{\rho_0\bm{u}} = 0$ for 
a constant $\rho_0(z)$), the impulse is
\begin{equation}
\bm{I} = \frac{1}{2}\int_{\mathcal{V}} \bm{r}\times(\nabla\times(\rho_0\bm{u}))d\bm{r},
\label{eqn:impulse}
\end{equation}
where $\bm{r} \equiv x\hat{x} + y\hat{y} + z\hat{z}$ is the position vector and $\mathcal{V}$ is a closed volume.
Basically our theory will consist of two parts:
\begin{enumerate}
\item An expectation for the time-derivative of the impulse of the thermal based on its
density excess (and/or other thermodynamic properties)
\item The time-derivative of an approximate expression for the thermal's impulse when it is
in its evolved vortex-ring state.
\end{enumerate}
Our expectation is that these two things will be approximately equal, and will tell us something
about the size of the thermal as it descends and thus its entrainment.

\subsection{Volumetric \& Surface terms of the impulse}
For the first part of our theory we'll need to find the time-derivative of the impulse. Before we do that, 
we need to break apart the impulse expression into surface terms and volumetric terms (as Daniel did in an email
on 9/4/2018). First, the component of the impulse in the $i$th direction for $i$ in $(x,y,z)$ is
$$
[\bm{r}\times(\grad\times(\rho_0\bm{u}))]_i = 
\grad_i(\rho_0\bm{u}) \cdot \bm{r}- (\bm{r}\cdot\grad)(\rho_0 u)_i = 
r_j\frac{\partial}{\partial x_i}(\rho_0 u)_j - r_j \frac{\partial}{\partial x_j}(\rho_0 u)_i.
$$
By definition, each of the RHS terms can be further expanded as
$$
r_j\frac{\partial}{\partial x_i}(\rho_0 u)_j =
\frac{\partial}{\partial x_i}(r_j(\rho_0 u)_j) - \delta_{ij}(\rho_0 u)_j,\qquad\text{and}
\qquad
r_j \frac{\partial}{\partial x_j}(\rho_0 u)_i =
\frac{\partial}{\partial x_j}(r_j(\rho_0 u)_i) - 3(\rho_0 u)_i.
$$
Plugging these expanded expressions in, we find that
$$
[\bm{r}\times(\grad\times(\rho_0\bm{u}))]_i = 
\frac{\partial}{\partial x_i}(r_j(\rho_0 u)_j) -
\frac{\partial}{\partial x_j}(r_j(\rho_0 u)_i) +
2(\rho_0 u)_i.
$$
Plugging back in to the full impulse defn in Eqn.~\ref{eqn:impulse}, we find
\begin{equation}
\bm{I} = \int_{\mathcal{V}} (\rho_0 \bm{u}) d\bm{r} + 
\frac{1}{2}\int_{\mathcal{V}} [\grad (\bm{r}\cdot(\rho_0 \bm{u})) - \grad \cdot(\bm{r}(\rho_0\bm{u}))] d\bm{r}.
\end{equation}
As this integral is being done over a closed volume, the last integral can be expressed in terms of surface terms,
\begin{equation}
\bm{I} = \int_{\mathcal{V}} (\rho_0 \bm{u}) d\bm{r} + 
\frac{1}{2}\int_{\mathcal{S}} d\bm{S}(\bm{r}\cdot(\rho_0 \bm{u})) - d\bm{S} \cdot(\bm{r}(\rho_0\bm{u})).
\label{eqn:surface_impulse}
\end{equation}

\section{Time derivative of the impulse}
The time derivative of the impulse is found from taking a time derivative of 
Eqn.~\ref{eqn:surface_impulse} and assuming that the fluid we're studying is unbounded
so that the surface terms don't change in time \citep[as in][]{shivamoggi2010},
\begin{equation}
\frac{d \bm{I}}{d t} = \int_{\mathcal{V}}\frac{\partial}{\partial t}(\rho_0 \bm{u}) d\bm{r}.
\end{equation}
(Aside: the unbounded assumption / the assumption that the surface integrals don't matter
seems a little sketchy to me for our sims, but it's common practice. Let's try it.
Also, I'm not entirely sure why $d/dt \rightarrow \partial/\partial t$
inside of the derivative, but oh well). 

By combining the non-viscous momentum and and anelastic continuity equations,
\begin{equation}
\begin{split}
\Div{\rho_0 \bm{u}} = 0, \\
\rho_0\frac{\partial \bm{u}}{\partial t} + \rho_0(\bm{u}\cdot\grad)\bm{u} =
-\grad P - \rho g \hat{z},
\end{split}
\end{equation}
where here I'm keeping the full $\rho$ on the buoyant term, we can find an 
equation for the expression in our impulse time derivative,
$$
\frac{\partial}{\partial t}{\rho_0 \bm{u}} + \grad\cdot(\rho_0\bm{u}\bm{u}) = -\grad P - (\rho_0 + \rho_1) g \hat{z}.
$$
Subtracting out hydrostatic balance, we have something like
$$
\frac{\partial}{\partial t}{\rho_0 \bm{u}} = - \grad\cdot(\rho_0\bm{u}\bm{u}) - \grad P_1 - \rho_1 g \hat{z}.
$$
and thus the change in impulse over time is
\begin{equation}
\frac{\partial \bm{I}}{\partial t} = 
-\int_{\mathcal{V}}\left(
\grad\cdot(\rho_0\bm{u}\bm{u}) + \grad P_1 + \rho_1 g \hat{z}
\right)d\bm{r}
= -\int_{\mathcal{V}}\rho_1 g\hat{z} d\bm{r} -
\int_{\mathcal{S}} [d\bm{S}\cdot(\rho_0\bm{u}\bm{u}) + d\bm{S}P_1].
\end{equation}
Generally \citep[in][and I think in other thermal work]{shivamoggi2010}, the
surface terms here are said to be zero (at least, under appropriate boundary
conditions), and so we're left with
\begin{equation}
\boxed{
\frac{d\bm{I}}{dt} = -gM_1 \hat{z}, \qquad M_1 \equiv \int_{\mathcal{V}}\rho_1 dV
}.
\label{eqn:dI_dt}
\end{equation}
Or, your change in impulse over time is directly proportional to the excess mass
in the thermal, or the integrated buoyant force over your thermal. 

\paragraph{Rough Estimate of $M_1$ size} 
For a thermal whose entropy perturbation is $\epsilon$, the temperature perturbation
is $T_{1, therm} = T_0(e^{\epsilon/c_P} - 1) \approx T_0\epsilon/c_P$, and the associated
density perturbation to maintain pressure equilibrium is 
$\rho_1 = P_0 / (T_0 + T_{1, therm}) - \rho_0$ where $\rho = \rho_0 + \rho_1$. For
$P_0 = \rho_0T_0$, this is $\rho_1 = \rho_0( [1 + T_{1, therm}]^{-1} - 1) \approx -\rho_0 \epsilon / c_P$.
So for positive $\epsilon$ (a hot thermal), we have a low density region, and for a falling
cold thermal (with negative $\epsilon$), we have a density excess. The total mass excess in the
thermal to start with is something like $M_1 \approx -\pi r_{therm}^2 \epsilon \rho_{0, therm} / c_P$,
assuming the density is fairly constant across the thermal. I'll do this calculation more
exactly at some point in the future, but this is a good rough estimate.

\section{Approximate expression for the impulse}
Going back to Eqn.~\ref{eqn:impulse}, we can see that the impulse can be simply expanded
into two expressions:
\begin{equation}
\bm{I} = \frac{1}{2}\int_{\mathcal{V}}\bm{r}\times(\grad\times(\rho_0\bm{u}))d\bm{r} 
= \frac{1}{2} \int_{\mathcal{V}} \left(\rho_0 \bm{r}\times \bm{\omega} + \bm{r} \times(\grad\rho_0 \times \bm{u}) \right) d\bm{r},
\end{equation}
where $\bm{\omega} = \grad\times\bm{u}$ is the vorticity.  The first term is identical to the term that shows up
in boussinesq flows, and \emph{under the assumption that $\rho$ is constant across the
thermal's depth} becomes \citep{saffman1970}
$$
\frac{1}{2} \int_{\mathcal{V}} \rho_0 \bm{r}\times \bm{\omega} d\bm{r} = \rho_0 c R^2 \Gamma \hat{z},
$$
where $R$ is the radius of the vortex ring, $\Gamma$ is its circulation, and $c$ is some constant describing its
aspect ratio. The second term here is
$$
\bm{r}\times(\grad\rho_0 \times\bm{u}) = (\bm{r}\cdot\bm{u})\grad\rho_0 - (\bm{r}\cdot\grad\rho_0)\bm{u}.
$$
Assuming that $\grad \rho_0 = \partial_z \rho_0 \hat{z}$, this becomes
$$
\bm{r}\times(\grad\rho_0 \times\bm{u}) = (-uz\hat{x} - vz\hat{y} + (xu + yv)\hat{z}).
$$
We don't care about the $x$ and $y$ terms here, because the main impulse of the thermal is vertical. We're essentially left with
$$
I_z = \rho_0 c R^2 \Gamma + \frac{1}{2}\int_{\mathcal{V}} u_{\perp} r_{\perp} \frac{\partial \rho_0}{\partial z} \,d\bm{r}.
$$
If the thermal is small, as we assumed in evaluating the first part, then
$\partial \rho_0 /\partial z$ is constant across the thermal, and the last term drops out
($u_{\perp} r_{\perp}$ at the bottom of the thermal cancels out with $u_{\perp} r_{\perp}$ at the top of the thermal
where it's oppositely signed).
If the thermal is large and the stratification of the background varies across the thermal's size, then we need to worry
about that second term.

For now, dropping the last term, we can say that for a small thermal, the time derivative of
the impulse should be roughly (assuming constant $c$)
$$
\frac{d I_z}{dt} = \frac{d}{dt}\left(\rho_0 c R^2 \Gamma\right) 
= c\left(R^2 \Gamma \frac{d\rho_0}{dt} + 2\Gamma \rho_0 R \frac{d R}{dt} + \rho_0 R^2 \frac{d\Gamma}{dt}\right)
$$
Note that $d\rho_0/dt = d\rho_0/dz \cdot dz/dt = w d\rho_0/dz$, and so
\begin{equation}
\boxed{
\frac{d I_z}{dt} = c\left(\Gamma R^2 w \frac{\partial \rho_0}{\partial z}\bigg|_{z=z_{th}}
+ 2 \rho_0 \Gamma R \frac{dR}{dt} + \rho_0 R^2 \frac{d\Gamma}{dt}\right)
}
\label{eqn:dIz_dt}
\end{equation}


\section{Putting it together: Theory for a small, anelastic thermal}
Basically the theory comes down to this: If this impulse framework is good for understanding
thermals, then Eqns.~\ref{eqn:dI_dt} \& \ref{eqn:dIz_dt} should be equal to each other.
In other words, we expect
\begin{equation}
-g M_1 \approx c\left(\Gamma R^2 w \frac{\partial \rho_0}{\partial z}\bigg|_{z=z_{th}}
+ 2 \rho_0 \Gamma R \frac{dR}{dt} + \rho_0 R^2 \frac{d\Gamma}{dt}\right).
\label{eqn:theory_result}
\end{equation}
This expression is slightly different from our first take at this theory. On the left,
we have mass instead of entropy. On the right, we're not assuming that the circulation
is constant.

\subsection{Thermal Growth predictions}
Now we're going to take the theory we derived above in Eqn.~\ref{eqn:theory_result}
and make it practical  We're going to make the following four assumptions:
\begin{enumerate}
\item \textbf{Assumption 1:} Circulation is constant in time, $\Gamma \neq \Gamma(t)$.
\item \textbf{Assumption 2:} The integrated mass anomaly (or entropy, depending on how the equations
are framed) is constant in time, $M_1 \neq M_1(t)$.
\item \textbf{Assumption 3:} The depth of the thermal follows a power-law in time,
$d(t) = L_z - z(t) \approx d_0 t^{\alpha}$. This also means that vertical velocity is defined
$w = \partial z / \partial t = - \partial d / \partial t = - d_0 \alpha t^{\alpha - 1}$.
\item \textbf{Assumption 4:} The radius of the thermal follows a different power-law in time,
$R(t) \approx R_0 t^{\beta}$.
\end{enumerate}
Under these assumptions, Eqn.~\ref{eqn:theory_result} becomes, after some rearranging,
\begin{equation}
C_1 \equiv -\frac{g M_1}{c \Gamma R_0^2}  = 
t^{2\beta - 1}\left(-d_0 t^{\alpha} \frac{\partial \rho_0}{\partial z} + 2 \rho_0 \right)
=
\rho_0 t^{2\beta - 1}\left(-d_0 t^{\alpha} \frac{\partial \ln \rho_0}{\partial z} + 2 \right)
\label{eqn:simple_theory}
\end{equation}
Now, for an adiabatic polytrope with adiabatic index $m_{ad} = (\gamma - 1)^{-1} = 1.5$ if $\gamma = 5/3$,
the local density of the thermal varies as
$$
\rho_0(t) = (1 + L_z - z(t))^{m_{ad}} = (1 + d(t))^{m_{ad}}, \qquad
\frac{\partial \ln \rho_0}{\partial z} = -m_{ad}\frac{1}{1 + d(t)}
$$

Let's look at this in two limits -- large stratification ($d \gg 1$) and small stratification ($d \rightarrow 0$). 
First, we examine large stratification. 
In this regime, it's more instructive to use the $\ln \rho_0$ form of
eqn. \ref{eqn:simple_theory}, where
\begin{equation}
\rho_0(t) \approx 
d(t)^{m_{ad}} \approx d_0^{m_{ad}} t^{m_{ad}\alpha},
\qquad
\frac{\partial \ln \rho_0}{\partial z} \approx 
-m_{ad} h^{-1}(t) \approx -(m_{ad}/d_0) t^{-\alpha}.
\end{equation}
Plugging these approximations in, we find
$$
C_1 \approx d_0^{m_{ad}} t^{m_{ad}\alpha + 2\beta - 1}(2 - m_{ad}).
$$
In order for this equation to be true, we need the time dependence to fall out of the
equation, so
\begin{equation}
m_{ad}\alpha + 2\beta - 1 = 0\,\,\rightarrow\,\, \beta = 0.5 - m_{ad} \alpha / 2.
\end{equation}
Thus, in the large stratification limit, we have a relation for how quickly the
radius scales compared to the depth.

For small stratification, the situation is muddier, but it's easier to use the $\rho_0$
and $\partial \rho_0 / \partial z$ form of eqn.~\ref{eqn:simple_theory}. Taylor expanding
$\rho_0$ around $d = 0$, we find
$$
\rho_0 \approx 1 + m_{ad} d + \cdots, \qquad 
\frac{\partial \rho_0}{\partial z} = -\frac{\partial \rho_0}{\partial d} \approx -(m_{ad} + \cdots) .
$$
After plugging in these approximations, we find
$$
C_1 \approx t^{2\beta - 1 + \alpha}(2 + 3m_{ad} d_0).
$$
Thus, in order for time dependence to fall out of this equation, we require
\begin{equation}
\alpha + 2\beta - 1 = 0\,\,\rightarrow\,\, \beta = 0.5 (1 -  \alpha).
\end{equation}
Which is interesting. In the boussinesq case, $\alpha = \beta = 0.5$, but here we see
that in the slightly (but importantly) stratified case, if $\alpha = 0.5$, $\beta = 0.25$.

 
One interesting result is that in both cases we find that
$$
\beta = 0.5 - X \alpha / 2,
$$
where $X = m_{ad}$ for the highly stratified case, and where maybe $X = 1$ for the minimally
stratified case. Regardless, the inclusion of stratification makes the scaling of the radius with
time less strong, and that scaling can be negative.

\bibliography{biblio.bib}
\end{document}
