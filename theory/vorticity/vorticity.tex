\documentclass[onecolumn, amsmath, amsfonts, amssymb]{aastex62}
\usepackage{mathtools}
\usepackage{natbib}
\usepackage{bm}
\newcommand{\vdag}{(v)^\dagger}
\newcommand\aastex{AAS\TeX}
\newcommand\latex{La\TeX}


\newcommand{\Div}[1]{\ensuremath{\nabla\cdot\left( #1\right)}}
\newcommand{\DivU}{\ensuremath{\nabla\cdot\bm{u}}}
\newcommand{\angles}[1]{\ensuremath{\left\langle #1 \right\rangle}}
\newcommand{\KS}[1]{\ensuremath{\text{KS}(#1)}}
\newcommand{\KSstat}[1]{\ensuremath{\overline{\text{KS}(#1)}}}
\newcommand{\grad}{\ensuremath{\nabla}}
\newcommand{\RB}{Rayleigh-B\'{e}nard }
\newcommand{\stressT}{\ensuremath{\bm{\bar{\bar{\Pi}}}}}
\newcommand{\lilstressT}{\ensuremath{\bm{\bar{\bar{\sigma}}}}}
\newcommand{\nrho}{\ensuremath{n_{\rho}}}
\newcommand{\approptoinn}[2]{\mathrel{\vcenter{
	\offinterlineskip\halign{\hfil$##$\cr
	#1\propto\cr\noalign{\kern2pt}#1\sim\cr\noalign{\kern-2pt}}}}}

\newcommand{\appropto}{\mathpalette\approptoinn\relax}
\newcommand{\pro}{\ensuremath{\text{Ro}_{\text{p}}}}
\newcommand{\con}{\ensuremath{\text{Ro}_{\text{c}}}}

\usepackage{color}
\newcommand{\gv}[1]{{\color{blue} #1}}

%% Tells LaTeX to search for image files in the 
%% current directory as well as in the figures/ folder.
\graphicspath{{./}{figs/}}

\begin{document}
\section{Vorticity arguments for entrainment}
\subsection{Quick Boussinesq argument}
The arguments in this section were inspired by Brett's paper draft, which Nadir sent me
around DFD time. Specifically, his arguments around "expansion by the creation of baroclinicity"
section.  He argues (in different terms, but mathematically the same)
that, for a boussinesq system with this momentum equation,
$$
\partial_t \bm{u}  + \grad\varpi - T_1 \hat{z} + \mathcal{R}\grad\times\bm{\omega} = -\bm{u}\times\bm{\omega},
$$
the system has a vorticity equation of the form,
\begin{equation}
\frac{D\bm{\omega}}{Dt} = -\frac{\partial T_1}{\partial r} + (\bm{\omega}\cdot\grad)\bm{u} + \mathcal{R}\grad^2\bm{\omega}.
\end{equation}
If we think of an azimuthally-symmetric thermal with no velocity in the $\hat{\phi}$ direction
and with vorticity only in the $\hat{\phi}$ direction, then we get, essentially,
\begin{equation}
\frac{D\omega_\phi}{Dt} = -\frac{\partial T_1}{\partial r} + \frac{u_r \omega_\phi}{r} + \mathcal{R}\grad^2\omega_\phi.
\end{equation}
Just based on some right-hand-rule arguments, $\omega_\phi$ will be positive in the vortex ring
core for an upflow (hot) thermal, and it will be negative for a downflow (cold) thermal.
In a upflow thermal, $\partial_r T_1$ is positive in the thermal center and negative at radii
larger than the vortex core's radius. In a downflow thermal, these signs are switched. This means
that $(-\partial_r T_1)$ \emph{destroys} vorticity for $r < r_{th}$ and \emph{creates} vorticity
for $r > r_{th}$ for both upflows and downflows. Thus, buoyancy creates entrainment through
the destruction of vorticity in the vortex core (and the creation of it further out).

\subsection{Expansion to stratification}
So that's the argument for the Boussinesq case. Which is great. What about the stratified
case?  The fully compressible momentum equation that we solve takes the form
$$
\frac{\partial\bm{u}}{\partial t} + \bm{u}\cdot\grad\bm{u} =
- \grad T - T \grad\ln\rho + \bm{g} + \text{viscous terms}.
$$
I'm going to drop the viscous terms for this analysis. Assuming $\bm{g} \equiv -\grad\phi$
for some potential $\phi$, if we take the curl of this equation to retrieve the vorticity equation,
we retrieve
$$
\frac{D\bm{\omega}}{D t} = % + (\bm{u}\cdot\grad)\bm{\omega} = 
(\bm{\omega}\cdot\grad)\bm{u} - \bm{\omega}(\grad\cdot\bm{u}) - \grad T \times \grad\ln\rho.
$$
So here we have a different-looking buoyancy term, and we've added back in the
$\bm{\omega}(\grad\cdot\bm{u})$ term because we're no longer incompressible. If we make the
same assumptions as above in the boussinesq case (azimuthally symmetric vortex ring), we find
$$
\frac{D\omega_\phi}{D t} = 
\frac{u_r\omega_\phi}{r} - \omega_\phi(\grad\cdot\bm{u}) - \grad T \times \grad\ln\rho.
$$
In the anelastic limit ($\partial_t \ln\rho = 0$, so $\ln_rho_1 = 0$), the continuity
equation can be expressed as $\grad\cdot\bm{u} = -\bm{u}\cdot\grad\ln\rho$. And, since the
density field isn't allowed to evolve and $\grad\ln\rho_0$ only varies in the $z$ direction
(the direction of stratification, we wind up with
$$
\frac{D\omega_\phi}{D t} = 
\frac{u_r\omega_\phi}{r} + u_z\omega_\phi\frac{\partial \ln\rho_0}{\partial z} - \grad T \times \grad\ln\rho.
$$
Assuming no variation in the phi direction, and assuming that $\grad\ln\rho = \grad\ln\rho_0$
and only has a z-component, the buoyancy term can be easily reduced to
$\grad T \times \grad\ln\rho_0 = -(\grad_r T_1)(\grad_z\ln\rho_0)\hat{\phi}$. Thus, the
ideal vorticity equation is
\begin{equation}
\frac{D \omega_\phi}{D t} = \frac{\partial T_1}{\partial r}\frac{\partial\ln\rho_0}{\partial z}
+ \frac{u_r\omega_\phi}{r} + u_z \omega_\phi \frac{\partial\ln\rho_0}{\partial z}.
\end{equation}
Since $\partial\ln\rho_0/\partial z$ is always negative, we can see that the first term, the
buoyancy term, has the same effect as in the Boussinesq case -- it causes entrainment by
destroying vorticity in the center of the thermal and adding it to the outside. The new term,
$u_z \omega_\phi (\partial\ln\rho_0/\partial z)$, however, requires some examination. We know
that the vertical velocity can be broken up into the velocity of the thermal frame
(let's call it $w_{th}$), and fluctuations around that, $u_z = w_{th} + w'$. Thus, we find
$$
u_z\omega_\phi = \omega_\phi w_{th} + \omega_\phi w'.
$$
For a downflowing (cold) thermal, $w_{th}$ and $\omega_\phi$ are both negative, so their
product is positive. For a upflowing thermal, both are positive and so is their product, so
the first term is positive definite. For a down (up) thermal, $w'$ is negative (positive) in the
thermal core, and positive (negative) outside of it. Thus, the second term is positive-definite
in the thermal core and negative-definite outside of it. To obtain the full stratification
term, we must multiply by $\partial_z\ln\rho_0$, which is negative everywhere.

We can thus re-write this equation as
\begin{equation}
\frac{D \omega_\phi}{D t} = -\frac{\partial T_1}{\partial r}\bigg|\frac{\partial\ln\rho_0}{\partial z}\bigg|
+ \frac{u_r\omega_\phi}{r} - \bigg| w_{th} \omega_\phi \frac{\partial\ln\rho_0}{\partial z}\bigg|
- w' \omega_\phi \bigg|\frac{\partial\ln\rho_0}{\partial z}\bigg|.
\end{equation}

\begin{table}
\label{table:signs}
\begin{center}
\begin{tabular}{ c c c }
\hline																	
Term	& Inside Thermal (cold / hot) 	& Outside Thermal (cold / hot)\\
\hline
$-\frac{\partial T_1}{\partial r}\bigg|\frac{\partial\ln\rho_0}{\partial z}\bigg|$ 	& + / - & - / + \\
$-\bigg| w_{th} \omega_\phi \frac{\partial\ln\rho_0}{\partial z}\bigg|$ 			& - / - & - / - \\
$-w' \omega_\phi \bigg|\frac{\partial\ln\rho_0}{\partial z}\bigg|$					& - / - & + / + \\	
\hline																	
\end{tabular}
\end{center}
\end{table}

So we can see that the third term is acting in an anti-buoyant manner for the cold thermal,
and a buoyant manner for the hot thermal. Cool.


\bibliography{biblio.bib}
\end{document}
