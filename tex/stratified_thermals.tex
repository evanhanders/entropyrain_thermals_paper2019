\documentclass[twocolumn, amsmath, amsfonts, amssymb, trackchanges]{aastex62}
\usepackage{mathtools}
\usepackage{natbib}
\usepackage{bm}
\newcommand{\vdag}{(v)^\dagger}
\newcommand\aastex{AAS\TeX}
\newcommand\latex{La\TeX}


\newcommand{\Div}[1]{\ensuremath{\nabla\cdot\left( #1\right)}}
\newcommand{\DivU}{\ensuremath{\nabla\cdot\bm{u}}}
\newcommand{\angles}[1]{\ensuremath{\left\langle #1 \right\rangle}}
\newcommand{\KS}[1]{\ensuremath{\text{KS}(#1)}}
\newcommand{\KSstat}[1]{\ensuremath{\overline{\text{KS}(#1)}}}
\newcommand{\grad}{\ensuremath{\nabla}}
\newcommand{\RB}{Rayleigh-B\'{e}nard }
\newcommand{\stressT}{\ensuremath{\bm{\bar{\bar{\Pi}}}}}
\newcommand{\lilstressT}{\ensuremath{\bm{\bar{\bar{\sigma}}}}}
\newcommand{\nrho}{\ensuremath{n_{\rho}}}
\newcommand{\approptoinn}[2]{\mathrel{\vcenter{
	\offinterlineskip\halign{\hfil$##$\cr
	#1\propto\cr\noalign{\kern2pt}#1\sim\cr\noalign{\kern-2pt}}}}}

\newcommand{\appropto}{\mathpalette\approptoinn\relax}
\newcommand{\pro}{\ensuremath{\text{Ro}_{\text{p}}}}
\newcommand{\con}{\ensuremath{\text{Ro}_{\text{c}}}}

\usepackage{color}
\newcommand{\gv}[1]{{\color{blue} #1}}

%% Tells LaTeX to search for image files in the 
%% current directory as well as in the figures/ folder.
\graphicspath{{./}{figs/}{../tex/figs/}}


\received{October 19, 2018}
\revised{\today}
\accepted{??}%\today}
\submitjournal{ApJ}

%%%%%%%%%%%%%%%%%%%%%%%%%%%%%%%%%%%%%%%%%%%%%%%%%%%%%%%%%%%%%%%%%%%%%%%%%%%%%%%
%% TITLE & AUTHORS
\shorttitle{Entrainment of thermals in stratified atmospheres}
\shortauthors{Anders et al.}

\begin{document}
\title{Entrainment of low Mach number thermals in stratified atmospheres}

\correspondingauthor{Evan H. Anders}
\email{evan.anders@colorado.edu}

\author[0000-0002-3433-4733]{Evan H. Anders}
\affil{Dept. Astrophysical \& Planetary Sciences, University of Colorado -- Boulder, Boulder, CO 80309, USA}
\affil{Laboratory for Atmospheric and Space Physics, Boulder, CO 80303, USA}
\author[0000-0002-7635-9728]{Daniel Lecoanet}
\affil{Stuff}
\author[0000-0001-8935-219X]{Benjamin P. Brown}
\affil{Dept. Astrophysical \& Planetary Sciences, University of Colorado -- Boulder, Boulder, CO 80309, USA}
\affil{Laboratory for Atmospheric and Space Physics, Boulder, CO 80303, USA}


\begin{abstract}
\end{abstract}

\keywords{hydrodynamics --- turbulence --- entrainment}

%%%%% Body of the paper
%%%%%%%%%%%%%%%%%%%%%%%%%%%%%%%%%%%%%%%%%%%%%%%%%%%%%%%%%%%%%%%%%%%%%
%% INTRODUCTION
\section{Introduction}
\label{sec:intro}

%%%%%%%%%%%%%%%%%%%%%%%%%%%%%%%%%%%%%%%%%%%%%%%%%%%%%%%%%%%%%%%%%%%%%%%%%%%%%%%
%% EXPERIMENT SECTION
\section{Experiment} 
\label{sec:experiment}

We perform direct numerical simulations of the evolution of dry thermals
in an ideal gas whose pressure ($P$), temperature ($T$), and density ($\rho$)
follow a nondimensional ideal gas law of $P = \rho T$.
We evolving the temperature, log density ($\ln\rho$), and velocity vector
($\bm{u} = u\hat{x} + v\hat{y} + z\hat{z}$) according to the fully compressible Navier-Stokes equations,
\begin{gather}
\frac{D \ln\rho}{Dt} = -\Div{\bm{u}}
	\label{eqn:continuity}
\\
\begin{align}
&\frac{D \bm{u}}{D t} =
-\grad T - T \grad\ln\rho + \bm{g} + \Div{\stressT}, 
\end{align}
	\label{eqn:full_momentum}
\\
\begin{align}
\frac{D T}{D t} + (\gamma - 1)T\Div{\bm{u}} + \frac{1}{\rho c_V}\Div{-\rho\chi\grad T} =\\
\frac{1}{\rho c_V}(\stressT \cdot \grad)\cdot \bm{u}, \nonumber
\end{align}
	\label{eqn:full_energy}
\end{gather}
where $\chi$ is the thermal diffusivity and with the viscous stress tensor given by
\begin{equation}
\Pi_{ij} \equiv \mu\left(\frac{\partial u_i}{\partial x_j} + 
\frac{\partial u_j}{\partial x_i} - \frac{2}{3}\delta_{ij}\grad\cdot\bm{u}\right),
	\label{eqn:stress_tensor}
\end{equation}
where $\delta_{ij}$ is the Kronecker delta and with $\mu = \rho\nu$ is the
dynamic viscosity and $\nu$ is the kinematic viscosity.
We split our thermodynamics into a time-stationary adiabatic background component
and a time-evolving fluctuation away from that background as
$T = T_0(z) + T_1(x, y, z, t)$ and $\ln\rho = \ln\rho_0(z) + \ln\rho_1(x, y, z, t)$.
We specify the background atmosphere as an adiabatic polytrope, as in
\cite{anders&brown2017}, so that
\begin{equation}
T_0(z) = (1 + L_z - z), \qquad \ln\rho_0(z) = m_{ad}\ln(T_0),
\end{equation}
where $L_z$ is the domain depth, and $m_{ad} = 1/(\gamma-1) = 1.5$ is the
adiabatic polytropic index. In the construction of these atmospheres, 
we assume that gravity $\bm{g} = -g\hat{z}$ is everywhere constant and that the
background atmosphere is in hydrostatic equilibrium such that
$\partial_z T_0 + T_0 \partial_z\ln\rho_0 - g = 0$. 


We assume a Pradtl number of
unity, $\text{Pr} = \nu/\chi = 1$, everywhere in the domain. We furthermore assume
that $\nu$ and $\chi$ are uniform everywhere in the domain, and that a negligibly
small diffusivity, $\chi_B = 0$ acts on the pure background terms in the thermal flux
such that
$$
\frac{1}{\rho_0 c_V}\grad\cdot(-\rho \chi_B \grad T) = 0.
$$
Under all of these assumptions, the equations we solve become
\begin{gather}
\frac{D \ln\rho_1}{Dt} = -\Div{\bm{u}} - w\partial_z\ln\rho_0
	\label{eqn:solved_continuity}
\\
\begin{align}
&\frac{D \bm{u}}{D t} =
-\grad T_1 - T_0 \grad\ln\rho_1 - T_1\partial_z\rho_0\hat{z} - T_1\grad\ln\rho_1 + \Div{\stressT}, 
\end{align}
	\label{eqn:solved_momentum}
\\
\begin{align}
&\frac{D T_1}{D t} + w\partial_z T_0 +  (\gamma - 1)(T_0 + T_1)\Div{\bm{u}} 
\\&- \frac{1}{c_V}\left\{\chi\left(\grad^2 T_1 + \partial_z\ln\rho_0\partial_z T_1 
+ \partial_z\ln\rho_1\partial_z T_0 + \grad\ln\rho_1 \cdot\grad T_1\right)  \right\}=\nonumber\\
&\frac{1}{\rho c_V}(\stressT \cdot \grad)\cdot \bm{u}, \nonumber
\end{align}
	\label{eqn:solved_energy}
\end{gather}


%%%%%%%%%%%%%%%%%%%%%%%%%%%%%%%%%%%%%%%%%%%%%%%%%%%%%%%%%%%%%%%%%%%%%%%%%%%%%%%
%% RESULTS SECTION
\section{Results}
\label{sec:results}

%%%%%%%%%%%%%%%%%%%%%%%%%%%%%%%%%%%%%%%%%%%%%%%%%%%%%%%%%%%%%%%%%%%%%%%%%%%%%%%
%% CONCLUSION SECTION
\section{Discussion}
\label{sec:discussion}

\begin{acknowledgements}
This work was supported by NASA Headquarters under the NASA Earth and Space
Science Fellowship Program -- Grant 80NSSC18K1199.
This work was additionally supported by  NASA LWS grant number NNX16AC92G.  
Computations were conducted 
with support by the NASA High End Computing (HEC) Program through the NASA 
Advanced Supercomputing (NAS) Division at Ames Research Center on Pleiades
with allocation GID s1647.
\end{acknowledgements}


\appendix
\section{Table of Simulations}
\label{appendix:table}

%\begin{deluxetable*}{c c c c c c c c c c c c c c c}
%\tabletypesize{\footnotesize}
%\caption{Table of simulation information
%\label{table:simulation_info}
%}
%\tablehead{																																															
%\colhead{Ra$_{\text{top}}$}	&	\colhead{Ta$_{\text{top}}$}	&	\colhead{Ro$_{\text{p, top}}$}	&	\colhead{Ra$_{\text{mid}}$}	&	\colhead{Ta$_{\text{mid}}$}	&	\colhead{Ro$_{\text{p, mid}}$}	&	\colhead{\con}	&	\colhead{$\mathcal{S}$}	&	\colhead{$L_x/L_z$}	&	\colhead{(nz,	nx,	ny)}	&					\colhead{Ro}      	&	\colhead{Re$_{\parallel}$} 	&	\colhead{Re$_{\perp}$}	&	\colhead{Nu}      	}														
%\startdata																																															
%\multicolumn{3}{l}{\textbf{Constant \pro$\,$, path III}}\\
%1.8	$\times 10^{	5	}$	&	4.1	$\times 10^{	7	}$	&	0.60	&	1.4	$\times 10^{	7	}$	&	3.0	$\times 10^{	9	}$	&	1.03	&	0.067	&	1.1	&	0.51	&	(256,		32,		32)	&	0.015	&	19.4	&	2.5	&	1.2	\\				
%1.2	$\times 10^{	9	}$	&	5.2	$\times 10^{	12	}$	&	0.60	&	9.2	$\times 10^{	10	}$	&	3.8	$\times 10^{	14	}$	&	1.03	&	0.015	&	3.0	&	0.07	&	(2048,		64,		64)	&	0.026	&	1771	&	32.0	&	15.4	\\				
%5.2	$\times 10^{	2	}$	&	4.6	$\times 10^{	3	}$	&	0.96	&	3.8	$\times 10^{	4	}$	&	3.4	$\times 10^{	5	}$	&	1.64	&	0.333	&	1.13	&	2.28	&	(64,		64,		64)	&	0.074	&	4.2	&	2.5	&	1.2	\\				
%3.8	$\times 10^{	8	}$	&	3.1	$\times 10^{	11	}$	&	0.96	&	2.8	$\times 10^{	10	}$	&	2.3	$\times 10^{	13	}$	&	1.64	&	0.035	&	6.0	&	0.12	&	(2048,		64,		64)	&	0.129	&	3906	&	113	&	66.9	\\				
%7.9	$\times 10^{	1	}$	&	1.0	$\times 10^{	2	}$	&	1.58	&	5.8	$\times 10^{	3	}$	&	7.4	$\times 10^{	3	}$	&	2.70	&	0.888	&	1.56	&	4.44	&	(64,		64,		64)	&	0.303	&	4.4	&	4.9	&	1.7	\\				
%1.4	$\times 10^{	7	}$	&	9.7	$\times 10^{	8	}$	&	1.58	&	1.0	$\times 10^{	9	}$	&	7.2	$\times 10^{	10	}$	&	2.70	&	0.119	&	10.0	&	0.30	&	(512,		128,		128)	&	0.376	&	1257	&	94.9	&	40.1	\\				
%\hline																																															
%\multicolumn{3}{l}{\textbf{Constant \con$\,$, path II}}\\
%8.6	$\times 10^{	4	}$	&	8.6	$\times 10^{	6	}$	&	0.74	&	6.3	$\times 10^{	6	}$	&	6.3	$\times 10^{	8	}$	&	1.26	&	0.1	&	1.47	&	0.68	&	(128,		128,		128)	&	0.051	&	40.2	&	6.7	&	2.2	\\				
%2.6	$\times 10^{	6	}$	&	2.6	$\times 10^{	8	}$	&	1.13	&	1.9	$\times 10^{	8	}$	&	1.9	$\times 10^{	10	}$	&	1.93	&	0.1	&	4.64	&	0.39	&	(256,		512,		512)	&	0.27	&	565	&	53.2	&	33.3	\\				
%1.4	$\times 10^{	3	}$	&	1.6	$\times 10^{	4	}$	&	1.01	&	1.1	$\times 10^{	5	}$	&	1.2	$\times 10^{	6	}$	&	1.72	&	0.3	&	1.47	&	1.90	&	(64,		128,		128)	&	0.124	&	11.3	&	5.4	&	1.8	\\				
%1.1	$\times 10^{	6	}$	&	1.2	$\times 10^{	7	}$	&	2.29	&	7.8	$\times 10^{	7	}$	&	8.6	$\times 10^{	8	}$	&	3.93	&	0.3	&	14.7	&	0.65	&	(192,		384,		384)	&	0.808	&	529	&	83.5	&	27.3	\\				
%5.5	$\times 10^{	1	}$	&	5.5	$\times 10^{	1	}$	&	1.65	&	4.0	$\times 10^{	3	}$	&	4.0	$\times 10^{	3	}$	&	2.82	&	1.0	&	1.47	&	4.84	&	(64,		128,		128)	&	0.303	&	3.6	&	4.4	&	1.5	\\				
%2.8	$\times 10^{	6	}$	&	2.8	$\times 10^{	6	}$	&	6.39	&	2.0	$\times 10^{	8	}$	&	2.0	$\times 10^{	8	}$	&	10.93	&	1.0	&	100	&	0.82	&	(256,		512,		512)	&	3.357	&	1099	&	220	&	46.6	\\				
%\hline																																															
%\multicolumn{3}{l}{\textbf{Constant $\mathcal{S}$, path I}}\\
%1.9	$\times 10^{	1	}$	&	1.1	$\times 10^{	-1	}$	&	10.00	&	1.4	$\times 10^{	3	}$	&	7.9	$\times 10^{		}$	&	17.12	&	13.2	&	2.0	&	9.91	&	(64,		64,		64)	&	3.668	&	3.0	&	10.2	&	1.7	\\				
%3.0	$\times 10^{	6	}$	&	1.1	$\times 10^{	9	}$	&	0.70	&	2.2	$\times 10^{	8	}$	&	8.1	$\times 10^{	10	}$	&	1.20	&	0.052	&	2.0	&	0.30	&	(512,		128,		128)	&	0.053	&	242	&	17.9	&	6.0	\\				
%3.0	$\times 10^{	1	}$	&	2.0	$\times 10^{	-1	}$	&	10.00	&	2.2	$\times 10^{	3	}$	&	1.5	$\times 10^{	1	}$	&	17.12	&	12.2	&	3.0	&	9.48	&	(64,		64,		64)	&	4.418	&	5.0	&	15.5	&	2.1	\\				
%1.3	$\times 10^{	7	}$	&	5.6	$\times 10^{	9	}$	&	0.80	&	9.7	$\times 10^{	8	}$	&	4.2	$\times 10^{	11	}$	&	1.37	&	0.048	&	3.0	&	0.23	&	(512,		128,		128)	&	0.08	&	592	&	33.4	&	13.6	\\				
%\enddata																																															
%\tablecomments{
%Input parameters and output parameters for select simulations are shown. For each of
%the eight paths in Fig. \ref{fig:parameter_space}b, we show information for the lowest and
%highest (Ra, Ta) point on that path. The first six rows show information for constant
%$\pro\,$ paths, the next six for constant \con$\,$ paths, and the last four for constant $\mathcal{S}$ paths.
%We show the input Ra, Ta, and \pro$\,$ at the top of the atmosphere, as well as their
%stratification-weighted values at the midplane of the atmosphere, which provide a more direct
%comparison to Boussinesq values \citep{unnoetall1960}. We also provide the input \con$\,$ at the
%top of the atmosphere, $\mathcal{S}$, aspect ratio ($L_x/L_z$), and coefficient resolution
%(nz, nx, ny). Each dimension of the physical grid is 3/2 the size of the coefficient grid 
%for adequate dealiasing of quadratic nonlinear terms.
%Output values of Ro, Re$_{\parallel}$, Re$_{\perp}$, and Nu are also provided.
%This table in its entirety is published as a supplemental \texttt{.csv} file with this
%manuscript and also online in a Zenodo repository \citep{supp_andersetall2019}.
%}
%\end{deluxetable*}
%


\bibliography{../tex/biblio.bib}

\listofchanges
\end{document}
