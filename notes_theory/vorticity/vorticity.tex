\documentclass[onecolumn, amsmath, amsfonts, amssymb]{aastex62}
\usepackage{mathtools}
\usepackage{natbib}
\usepackage{bm}
\newcommand{\vdag}{(v)^\dagger}
\newcommand\aastex{AAS\TeX}
\newcommand\latex{La\TeX}


\newcommand{\Div}[1]{\ensuremath{\nabla\cdot\left( #1\right)}}
\newcommand{\DivU}{\ensuremath{\nabla\cdot\bm{u}}}
\newcommand{\angles}[1]{\ensuremath{\left\langle #1 \right\rangle}}
\newcommand{\KS}[1]{\ensuremath{\text{KS}(#1)}}
\newcommand{\KSstat}[1]{\ensuremath{\overline{\text{KS}(#1)}}}
\newcommand{\grad}{\ensuremath{\nabla}}
\newcommand{\RB}{Rayleigh-B\'{e}nard }
\newcommand{\stressT}{\ensuremath{\bm{\bar{\bar{\Pi}}}}}
\newcommand{\lilstressT}{\ensuremath{\bm{\bar{\bar{\sigma}}}}}
\newcommand{\nrho}{\ensuremath{n_{\rho}}}
\newcommand{\approptoinn}[2]{\mathrel{\vcenter{
	\offinterlineskip\halign{\hfil$##$\cr
	#1\propto\cr\noalign{\kern2pt}#1\sim\cr\noalign{\kern-2pt}}}}}

\newcommand{\appropto}{\mathpalette\approptoinn\relax}
\newcommand{\pro}{\ensuremath{\text{Ro}_{\text{p}}}}
\newcommand{\con}{\ensuremath{\text{Ro}_{\text{c}}}}

\usepackage{color}
\newcommand{\gv}[1]{{\color{blue} #1}}

%% Tells LaTeX to search for image files in the 
%% current directory as well as in the figures/ folder.
\graphicspath{{./}{figs/}}

\begin{document}
\section{Vorticity arguments for entrainment}
\subsection{Quick Boussinesq argument}
The arguments in this section were inspired by Brett's paper draft, which Nadir sent me
around DFD time. Specifically, his arguments around "expansion by the creation of baroclinicity"
section.  He argues (in different terms, but mathematically the same)
that, for a boussinesq system with this momentum equation,
$$
\partial_t \bm{u}  + \grad\varpi - T_1 \hat{z} + \mathcal{R}\grad\times\bm{\omega} = -\bm{u}\times\bm{\omega},
$$
the system has a vorticity equation of the form,
\begin{equation}
\frac{D\bm{\omega}}{Dt} = -\frac{\partial T_1}{\partial r} + (\bm{\omega}\cdot\grad)\bm{u} + \mathcal{R}\grad^2\bm{\omega}.
\end{equation}
If we think of an azimuthally-symmetric thermal with no velocity in the $\hat{\phi}$ direction
and with vorticity only in the $\hat{\phi}$ direction, then we get, essentially,
\begin{equation}
\frac{D\omega_\phi}{Dt} = -\frac{\partial T_1}{\partial r} + \frac{u_r \omega_\phi}{r} + \mathcal{R}\grad^2\omega_\phi.
\end{equation}
Just based on some right-hand-rule arguments, $\omega_\phi$ will be positive in the vortex ring
core for an upflow (hot) thermal, and it will be negative for a downflow (cold) thermal.
In a upflow thermal, $\partial_r T_1$ is positive in the thermal center and negative at radii
larger than the vortex core's radius. In a downflow thermal, these signs are switched. This means
that $(-\partial_r T_1)$ \emph{destroys} vorticity for $r < r_{th}$ and \emph{creates} vorticity
for $r > r_{th}$ for both upflows and downflows. Thus, buoyancy creates entrainment through
the destruction of vorticity for $r < r_{th}$ (and the creation of it further out).

\subsection{Expansion to stratification}
So that's the argument for the Boussinesq case. Which is great. What about the stratified
case?  The fully compressible momentum equation that we solve takes the form
$$
\frac{\partial\bm{u}}{\partial t} + \bm{u}\cdot\grad\bm{u} =
- \grad T - T \grad\ln\rho + \bm{g} + \text{viscous terms}.
$$
I'm going to drop the viscous terms for this analysis. Assuming $\bm{g} \equiv -\grad\phi$
for some potential $\phi$, if we take the curl of this equation to retrieve the vorticity equation,
we retrieve
$$
\frac{D\bm{\omega}}{D t} = % + (\bm{u}\cdot\grad)\bm{\omega} = 
(\bm{\omega}\cdot\grad)\bm{u} - \bm{\omega}(\grad\cdot\bm{u}) - \grad T \times \grad\ln\rho.
$$
So here we have a different-looking buoyancy term, and we've added back in the
$\bm{\omega}(\grad\cdot\bm{u})$ term because we're no longer incompressible. If we make the
same assumptions as above in the boussinesq case (azimuthally symmetric vortex ring), we find
$$
\frac{D\omega_\phi}{D t} = 
\frac{u_r\omega_\phi}{r} - \omega_\phi(\grad\cdot\bm{u}) - \grad T \times \grad\ln\rho.
$$
In the anelastic limit ($\partial_t \ln\rho = 0$, so $\ln{\rho_1} = 0$), the continuity
equation can be expressed as $\grad\cdot\bm{u} = -\bm{u}\cdot\grad\ln\rho$. And, since the
density field isn't allowed to evolve and $\grad\ln\rho$ only varies in the $z$ direction
(the direction of stratification, we wind up with
$$
\frac{D\omega_\phi}{D t} = 
\frac{u_r\omega_\phi}{r} + u_z\omega_\phi\frac{\partial \ln\rho_0}{\partial z} - \grad T \times \grad\ln\rho.
$$
Assuming no variation in the phi direction, and assuming that $\grad\ln\rho = \grad\ln\rho_0$
and only has a z-component, the buoyancy term can be easily reduced to
$\grad T \times \grad\ln\rho_0 = -(\grad_r T_1)(\grad_z\ln\rho_0)\hat{\phi}$. Thus, the
ideal vorticity equation is
\begin{equation}
\frac{D \omega_\phi}{D t} = \frac{\partial T_1}{\partial r}\frac{\partial\ln\rho_0}{\partial z}
+ \frac{u_r\omega_\phi}{r} + u_z \omega_\phi \frac{\partial\ln\rho_0}{\partial z}.
\end{equation}
Since $\partial\ln\rho_0/\partial z$ is always negative, we can see that the first term, the
buoyancy term, has the same effect as in the Boussinesq case -- it causes entrainment by
destroying vorticity in the center of the thermal and adding it to the outside. The new term,
$u_z \omega_\phi (\partial\ln\rho_0/\partial z)$, however, requires some examination. We know
that the vertical velocity can be broken up into the velocity of the thermal frame
(let's call it $w_{th}$), and fluctuations around that, $u_z = w_{th} + w'$. Thus, we find
$$
u_z\omega_\phi = \omega_\phi w_{th} + \omega_\phi w'.
$$
For a downflowing (cold) thermal, $w_{th}$ and $\omega_\phi$ are both negative, so their
product is positive. For a upflowing thermal, both are positive and so is their product, so
the first term is positive definite. For a down (up) thermal, $w'$ is negative (positive) in the
thermal core, and positive (negative) outside of it. Thus, the second term is positive-definite
in the thermal core and negative-definite outside of it. To obtain the full stratification
term, we must multiply by $\partial_z\ln\rho_0$, which is negative everywhere.

We can thus re-write this equation as
\begin{equation}
\frac{D \omega_\phi}{D t} = -\frac{\partial T_1}{\partial r}\bigg|\frac{\partial\ln\rho_0}{\partial z}\bigg|
+ \frac{u_r\omega_\phi}{r} - \bigg| w_{th} \omega_\phi \frac{\partial\ln\rho_0}{\partial z}\bigg|
- w' \omega_\phi \bigg|\frac{\partial\ln\rho_0}{\partial z}\bigg|.
\label{eqn:stratified_vorticity}
\end{equation}
So we have one buoyancy term and ``two'' new stratification-related terms in the vorticity
equation. We examine their signs in Table \ref{table:signs}, and we see that, for a downflowing
thermal, the last term in Eqn.~\ref{eqn:stratified_vorticity} acts in an anti-buoyant manner
and should reduce entrainment. The implications of these new terms on circulation should be examined
(but I ran out of time to do that carefully here).

\begin{table}[h!]
\caption{An examination of the signs of the terms in Eqn.~\ref{eqn:stratified_vorticity}.
Signs are given for the terms at radii smaller than the vorticity peak ($r < r_{th}$,
``inside thermal'') and
also radii larger than that peak ($r > r_{th}$, ``outside thermal''). The sign of the
term is given first for a cold thermal, then for a hot thermal (e.g., ``+/-'' means that
the term is positive for a cold thermal and negative for a hot one).}
\label{table:signs}
\begin{center}
\begin{tabular}{ c c c }
\hline																	
Term	& Inside Thermal (cold / hot) 	& Outside Thermal (cold / hot)\\
\hline
$-\frac{\partial T_1}{\partial r}\bigg|\frac{\partial\ln\rho_0}{\partial z}\bigg|$ 	& + / - & - / + \\
$-\bigg| w_{th} \omega_\phi \frac{\partial\ln\rho_0}{\partial z}\bigg|$ 			& - / - & - / - \\
$-w' \omega_\phi \bigg|\frac{\partial\ln\rho_0}{\partial z}\bigg|$					& - / - & + / + \\	
\hline																	
\end{tabular}
\end{center}
\end{table}

\section{Circulation Equation}
So we've seen that the \emph{vorticity} equation gains a new term that can help
phenomenologically explain the effects of stratification on a thermal. Now we turn to
the circulation equation. To get there, we go through the normal steps of Kelvin's
circulation theorem. We start off with the definition of the time derivative of circulation
\begin{equation}
\frac{d\Gamma}{dt} = \frac{d}{dt} \oint \bm{u}\cdot d\bm{x} 
= \oint \frac{d \bm{u}}{dt}\cdot d\bm{x} + \oint \bm{u}\cdot\frac{d}{dt}d\bm{x}.
\end{equation}
In the last term in the above equation, if we think about a piece of a line element
$d\bm{x}$ connecting two fluid parcels, the edges of the line are stretched by the
difference in velocity at its two endpoints, $d( d\bm{x})dt = d\bm{u}$. Plugging this
into the above integral (in the limit $d\bm{x} \rightarrow 0$),
$$
\oint\bm{u}\cdot\frac{d}{dt}d\bm{x} = \oint \bm{u}\cdot d\bm{u}
= \oint d|\bm{u}|^2 / 2 = 0,
$$
and this integral is zero along a closed path. Thus, the change in time of circulation
is simply
\begin{equation}
\frac{d\Gamma}{dt} = \oint \frac{d\bm{u}}{dt} \cdot d\bm{x}.
\end{equation}
At this point we specify that the velocity we care about, $\bm{u}$, is the velocity
in the inertial frame, and so
\begin{equation}
\frac{d \bm{u}}{dt} = \frac{\partial \bm{u}}{\partial t} + \bm{u}\cdot\grad\bm{u}
= -\grad T - T \grad\ln\rho + \bm{g} + \grad\cdot\stressT.
\end{equation}
In other words, the RHS of the moment equation gets substituted directly into the integral,
$$
\frac{d\Gamma}{dt} = \oint \left[\left( - T \grad\ln\rho + \grad\cdot\stressT\right) -
\left(\grad T - \grad \phi\right)\right] \cdot d\bm{x},
$$
where $\bm{g} \equiv -\grad\phi$ is defined by its potential. The last two terms in this 
integral are conservative forces, and their integral around a closed loop is zero. Thus,
we have
\begin{equation}
\frac{d\Gamma}{dt} = -\oint\left[T_0\grad\ln\rho_0 + T_1\grad\ln\rho_0 + T_0\grad\ln\rho_1 +
T_1\grad\ln\rho_1\right]\cdot d\bm{x}
+ \oint (\grad\cdot\stressT)\cdot d\bm{x},
\end{equation}
Let's go ahead and ignore the stress tensor for now and use
Stokes' theorem, and make some of the assumptions we did earlier (azimuthal symmetry,
only velocity in $(\hat{r}, \hat{z})$ directions, background is only stratified in 
$\hat{z}$ direction). We arrive at
\begin{equation}
\frac{d\Gamma}{dt} = \iint\left( \frac{\partial T_1}{\partial r}\frac{\partial\ln\rho_0}{\partial z}
+ \frac{\partial T_0}{\partial z}\frac{\partial \ln\rho_1}{\partial r} 
+ \frac{\partial T_1}{\partial z}\frac{\partial \ln\rho_1}{\partial r} 
- \frac{\partial T_1}{\partial r}\frac{\partial \ln\rho_1}{\partial z}\right)
\cdot d\bm{A}
\end{equation}
In our non-dimensionalized system, $\partial_z T_0 = -1$, and 
$\partial_z \ln\rho_0 = -m_{ad}/T_0$, so we can rewrite
\begin{equation}
\frac{d\Gamma}{dt} = -\iint\left( \frac{m_{ad}}{T_0}\frac{\partial T_1}{\partial r}
+ \frac{\partial \ln\rho_1}{\partial r} 
+ \frac{\partial T_1}{\partial z}\frac{\partial \ln\rho_1}{\partial r} 
- \frac{\partial T_1}{\partial r}\frac{\partial \ln\rho_1}{\partial z}\right)
\cdot d\bm{A}.
\end{equation}
For a cold thermal, $\ln\rho_1 > 0$ and $T_1 < 0$, so the signs of their r-derivatives will be
opposite one another, and the two terms in the integral will work against one another. Furthermore,
the r-derivatives will have different signs near the axis of symmetry and far from it, so likely
\emph{both of the first two terms cancel out entirely}. We're left with
\begin{equation}
\frac{d\Gamma}{dt} = -\iint\left( 
\frac{\partial T_1}{\partial z}\frac{\partial \ln\rho_1}{\partial r} 
- \frac{\partial T_1}{\partial r}\frac{\partial \ln\rho_1}{\partial z}\right)
\cdot d\bm{A}.
\end{equation}
I think for similar reasons that the first two terms disappeared, these terms will also
disappear. If you break up the thermal into four ``quadrants'' (upper left, upper right,
lower left, lower right) and examine the sign of these terms in each of those, we also see
that they work against one another. If buoyancy isn't important, 
we'll be left with just the viscous terms,
\begin{equation}
\frac{d\Gamma}{dt} = \oint(\grad\cdot\stressT)\cdot d\bm{x},
\end{equation}
But *maybe* those buoyancy
terms aren't as perfectly symmetric as we think, and they should be measured.

\textbf{What we've learned:} The buoyancy term shows up in a more complex manner in the
equation for $d\Gamma/dt$, but it probably goes away due to symmetry arguments. I'll need
to take some measurements from the DNS to know for sure.


\bibliography{biblio.bib}
\end{document}
