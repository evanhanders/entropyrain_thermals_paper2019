\documentclass[onecolumn, amsmath, amsfonts, amssymb]{aastex62}
\usepackage{mathtools}
\usepackage{natbib}
\usepackage{bm}
\newcommand{\vdag}{(v)^\dagger}
\newcommand\aastex{AAS\TeX}
\newcommand\latex{La\TeX}


\newcommand{\Div}[1]{\ensuremath{\nabla\cdot\left( #1\right)}}
\newcommand{\DivU}{\ensuremath{\nabla\cdot\bm{u}}}
\newcommand{\angles}[1]{\ensuremath{\left\langle #1 \right\rangle}}
\newcommand{\KS}[1]{\ensuremath{\text{KS}(#1)}}
\newcommand{\KSstat}[1]{\ensuremath{\overline{\text{KS}(#1)}}}
\newcommand{\grad}{\ensuremath{\nabla}}
\newcommand{\RB}{Rayleigh-B\'{e}nard }
\newcommand{\stressT}{\ensuremath{\bm{\bar{\bar{\Pi}}}}}
\newcommand{\lilstressT}{\ensuremath{\bm{\bar{\bar{\sigma}}}}}
\newcommand{\nrho}{\ensuremath{n_{\rho}}}
\newcommand{\approptoinn}[2]{\mathrel{\vcenter{
	\offinterlineskip\halign{\hfil$##$\cr
	#1\propto\cr\noalign{\kern2pt}#1\sim\cr\noalign{\kern-2pt}}}}}

\newcommand{\appropto}{\mathpalette\approptoinn\relax}
\newcommand{\pro}{\ensuremath{\text{Ro}_{\text{p}}}}
\newcommand{\con}{\ensuremath{\text{Ro}_{\text{c}}}}

\usepackage{color}
\newcommand{\gv}[1]{{\color{blue} #1}}

%% Tells LaTeX to search for image files in the 
%% current directory as well as in the figures/ folder.
\graphicspath{{./}{figs/}}

\begin{document}
\section{Description of stratified thermals}
In the thermal simulations, we have nine variables, although many of them are geometric and
likely intrinsically linked. These variables are:
\begin{enumerate}
\item $B$, the integrated total entropy leading to buoyancy
\item $V$, the thermal volume
\item $R$, the radius of the thermal
\item $r$, the radius (from axis of symmetry) of the thermal's vortex torus.
\item $R_z$, the thermal's radius in the z-direction (we assume it is an oblate spheroid)
\item $w$, the thermal bulk vertical velocity
\item $z$, the height of the thermal
\item $\Gamma$, the circulation of the vortex ring
\item $\rho$, the local density at the height of the thermal.
\end{enumerate}
From the simulations we've done so far, when viscous heating and detrainment are neglected, and the
diffusivities are sufficiently low (Re is sufficiently high) such that diffusive terms are not dominant, the following
equations seem to govern the thermal evolution:
\begin{gather}
B = \int \rho s_1 dV = \text{const}, \\
R_z = R/A,\qquad \text{ where $A$ is a constant aspect ratio },\\
r = f R, \qquad \text{ where $f$ is a constant fraction }, \\
V = (4/3)\pi h R^2 = (4/3)\pi R^3/A = V_0 R^3 = (V_0/f^3) r^3, \\
w = \frac{\partial z}{\partial t},\\
\rho = [1 + (\grad T)_{ad}(z - F^{-1})]^{m_{ad}} = T^{m_{ad}}, \text{(where $F = L_{thermal}/L_{box}$)}\\
\rho V w = B t + M_0,  \label{eqn:momentum} \\
\frac{1}{2}\rho r^2 \Gamma = B t + I_0, \label{eqn:impulse} \\
\Gamma \sim \text{const.}
\end{gather}
Note that all of these equations are azimuthal averages, I think. You'd probably have to
multiply by $2\pi$ to retrieve the ``full'' values in a 3D sim.

\section{Solve it out}
One thing that we have here that we didn't have a month ago is the constant offsets in
the impulse and momentum: $I_0$ and $M_0$.  They make it so that plugging the two together
is less straightforward that we had once thought, but it's still possible. From 
Eqn.~\ref{eqn:impulse}, we solve for
\begin{equation}
r = \sqrt{2\frac{B t + I_0}{\rho\Gamma}}.
\end{equation}
Plugging this in to Eqn.~\ref{eqn:momentum}, we get
\begin{equation}
\rho V w = \left(\frac{V_0}{f^3}\right)\rho r^3 w
= \left(\frac{V_0}{f^3}\right)\rho w \left(2\frac{B t + I_0}{\rho\Gamma}\right)^{3/2} = Bt + M_0,
\end{equation}
or, with $w = dz/dt$,
\begin{equation}
\rho^{-1/2}dz = \left(\frac{f^3 \Gamma^{3/2}}{2^{3/2}}\right)\frac{Bt + M_0}{(Bt + I_0)^{3/2}}dt
\end{equation}
But we know that $\rho = T^{m_{ad}},$ with $T$ the temperature, and $dT = (\grad T)_{ad} dz$,
and we can define $\tau = Bt + I_0$, with $d\tau = B dt$. Substituting these in, we retrieve
\begin{equation}
T^{-m_{ad}/2}dT 
= \left(\frac{f^3 \Gamma^{3/2}(\grad T)_{ad}}{2^{3/2} B}\right)\frac{\tau + M_0 - I_0}{\tau^{3/2}}d\tau.
\end{equation}
Defining a constant $C \equiv [f^3 \Gamma(\grad T)_{ad}]/[2^{3/2} B]$, we can write this more simply,
\begin{equation}
T^{-m_{ad}/2}dT = C\sqrt{\Gamma}\left(\frac{d\tau}{\tau^{1/2}} + \frac{M_0 - I_0}{\tau^{3/2}}d\tau\right).
\end{equation}
The choice to leave $\sqrt{\Gamma}$ outside of $C$ occurs because the value of $\tau$ is likely
negative for a downward thermal, and combining the $\sqrt{\Gamma}$ with $\tau$ keeps the 
numbers finite. On the bright side, these are all just power-law integrals, and with our choice 
of $m_{ad} = 3/2$, we retrieve
\begin{equation}
4 T\bigg|_{T(t=0)}^{T} 
= 2C\sqrt{\Gamma}\left(\tau^{1/2} - \frac{M_0 - I_0}{\tau^{1/2}}\right)\bigg|_{\tau(t=0)}^{\tau}
\end{equation}
Plugging things in and rearranging, our final equation for the evolution of temperature with time
is
\begin{equation}
T(t) = \left[\frac{C\sqrt{\Gamma}}{2}
\left(\sqrt{Bt + I_0} + \frac{I_0 - M_0}{\sqrt{Bt + I_0}} - 2\sqrt{I_0} + \frac{M_0}{\sqrt{I_0}}\right)
+ T(t=0)^{1/4}\right]^{4},
\end{equation}
where the thermal velocity can be simply retrieved as just
\begin{equation}
w = (\grad T)_{ad}^{-1}\frac{\partial T}{\partial t},
\end{equation}
where, for completeness, we solve out
\begin{equation}
\frac{\partial T}{\partial t} 
= B C \sqrt{\Gamma}( T^{3/4} ) \left(\frac{1}{(B t + I_0)^{1/2}} + \frac{M_0 - I_0}{(B t + I_0)^{3/2}}\right).
\end{equation}


\bibliography{biblio.bib}
\end{document}
