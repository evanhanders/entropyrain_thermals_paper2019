\documentclass[onecolumn, amsmath, amsfonts, amssymb]{aastex62}
\usepackage{mathtools}
\usepackage{natbib}
\usepackage{bm}
\newcommand{\vdag}{(v)^\dagger}
\newcommand\aastex{AAS\TeX}
\newcommand\latex{La\TeX}


\newcommand{\Div}[1]{\ensuremath{\nabla\cdot\left( #1\right)}}
\newcommand{\DivU}{\ensuremath{\nabla\cdot\bm{u}}}
\newcommand{\angles}[1]{\ensuremath{\left\langle #1 \right\rangle}}
\newcommand{\KS}[1]{\ensuremath{\text{KS}(#1)}}
\newcommand{\KSstat}[1]{\ensuremath{\overline{\text{KS}(#1)}}}
\newcommand{\grad}{\ensuremath{\nabla}}
\newcommand{\RB}{Rayleigh-B\'{e}nard }
\newcommand{\stressT}{\ensuremath{\bm{\bar{\bar{\Pi}}}}}
\newcommand{\lilstressT}{\ensuremath{\bm{\bar{\bar{\sigma}}}}}
\newcommand{\nrho}{\ensuremath{n_{\rho}}}
\newcommand{\approptoinn}[2]{\mathrel{\vcenter{
	\offinterlineskip\halign{\hfil$##$\cr
	#1\propto\cr\noalign{\kern2pt}#1\sim\cr\noalign{\kern-2pt}}}}}

\newcommand{\appropto}{\mathpalette\approptoinn\relax}
\newcommand{\pro}{\ensuremath{\text{Ro}_{\text{p}}}}
\newcommand{\con}{\ensuremath{\text{Ro}_{\text{c}}}}

\usepackage{color}
\newcommand{\gv}[1]{{\color{blue} #1}}

%% Tells LaTeX to search for image files in the 
%% current directory as well as in the figures/ folder.
\graphicspath{{./}{figs/}}

\begin{document}
\section{Description of stratified thermals}
In the thermal simulations, we have nine variables, although many of them are geometric and
likely intrinsically linked. These variables are:
\begin{enumerate}
\item $B$, the integrated total entropy leading to buoyancy
\item $V$, the thermal volume
\item $R$, the radius of the thermal
\item $r$, the radius (from axis of symmetry) of the thermal's vortex torus.
\item $R_z$, the thermal's radius in the z-direction (we assume it is an oblate spheroid)
\item $w$, the thermal bulk vertical velocity
\item $z$, the height of the thermal
\item $\Gamma$, the circulation of the vortex ring
\item $\rho$, the local density at the height of the thermal.
\end{enumerate}
From the simulations we've done so far, when viscous heating and detrainment are neglected, and the
diffusivities are sufficiently low (Re is sufficiently high) such that diffusive terms are not dominant, the following
equations seem to govern the thermal evolution:
\begin{gather}
B = \int \rho s_1 dV = 2\pi \iint \rho s_1 r dr dz = \text{const}, \\
R_z = R/A,\qquad \text{ where $A$ is a constant aspect ratio },\\
r = f R, \qquad \text{ where $f$ is a constant fraction }, \\
V = (4/3)\pi h R^2 = (4/3)\pi R^3/A = V_0 R^3 = (V_0/f^3) r^3, \\
w = \frac{\partial z}{\partial t},\\
\rho = [1 + (\grad T)_{ad}(z - F^{-1})]^{m_{ad}} = T^{m_{ad}}, \text{(where $F = L_{thermal}/L_{box}$)}\\
\rho V w = \beta B t + M_0,  \label{eqn:momentum} \\
\pi\rho r^2 \Gamma = B t + I_0, \label{eqn:impulse} \\
\Gamma \sim \text{const.}
\end{gather}

\section{Solution for Temperature, height, and velocity}
One thing that we have here that we didn't have a month ago is the constant offsets in
the impulse and momentum: $I_0$ and $M_0$.  They make it so that plugging the two together
is less straightforward that we had once thought, but it's still possible. From 
Eqn.~\ref{eqn:impulse}, we solve for
\begin{equation}
r = \sqrt{\frac{B t + I_0}{\pi\rho\Gamma}}.
\end{equation}
Plugging this in to Eqn.~\ref{eqn:momentum}, we get
\begin{equation}
\rho V w = \left(\frac{V_0}{f^3}\right)\rho r^3 w
= \left(\frac{V_0}{f^3}\right)\rho w \left(\frac{B t + I_0}{\pi\rho\Gamma}\right)^{3/2} = \beta Bt + M_0,
\end{equation}
or, with $w = dz/dt$,
\begin{equation}
\rho^{-1/2}dz = \left(\frac{f^3 (\pi\Gamma)^{3/2}}{V_0}\right)\frac{\beta Bt + M_0}{(Bt + I_0)^{3/2}}dt
\end{equation}
But we know that $\rho = T^{m_{ad}},$ with $T$ the temperature, and $dT = (\grad T)_{ad} dz$,
and we can define $\tau = (Bt + I_0)/\Gamma$, with $d\tau = (B/\Gamma) dt$ (these choice of constants
ensure that $\tau$ is positive definite and monotonically increases). Substituting these in, we retrieve
\begin{equation}
T^{-m_{ad}/2}dT 
= \left(\frac{\pi^{3/2}f^3 \Gamma(\grad T)_{ad}}{V_0 B}\right)\frac{\beta\Gamma \tau + M_0 - \beta I_0}{\tau^{3/2}}d\tau.
\end{equation}
Defining a constant $C \equiv [\pi^{3/2} f^3 \Gamma(\grad T)_{ad}]/[V_0 B]$, we can write this more simply,
\begin{equation}
T^{-m_{ad}/2}dT = C\left(\beta \Gamma\tau^{-1/2}d\tau + (M_0 - \beta I_0)\tau^{-3/2}d\tau\right).
\label{eqn:integrand}
\end{equation}
These are all just power-law integrals, and so long as
we avoid the special case where $m_{ad} = 2$ (the left-hand integral is a log then), we retrieve
\begin{equation}
\frac{1}{1 - m_{ad}/2} T^{1 - m_{ad}/2}\bigg|_{T_0}^{T} 
= 2C\left(\beta\Gamma \tau^{1/2} - (M_0 - \beta I_0)\tau^{-1/2}\right)\bigg|_{\tau_0}^{\tau}
\end{equation}
For simplicity of writing, I define $\alpha^{-1} =  1 - m_{ad}/2$ and 
$$
\xi(\tau) = \beta\Gamma \tau^{1/2} - (M_0 - \beta I_0)\tau^{-1/2}.
$$
Plugging things in and rearranging, our final equation for the evolution of temperature with time
is
\begin{equation}
T(t) = \left[\frac{2C}{\alpha}(\xi[\tau] - \xi[\tau_0])
+ T_0^{1/\alpha}\right]^{\alpha},
\end{equation}
where the thermal velocity and height can be simply retrieved as just
\begin{equation}
w = (\grad T)_{ad}^{-1}\frac{\partial T}{\partial t}, \qquad z = (\grad T)_{ad}^{-1}(T - 1) + F^{-1}
\end{equation}
where, for completeness, we solve out
\begin{equation}
\frac{\partial T}{\partial t} 
= C \frac{B}{\Gamma}( T^{m_{ad}/2} ) \left(\beta\Gamma\tau^{-1/2} + (M_0 - \beta I_0)\tau^{-3/2}\right)
.
\end{equation}
In our simulations where $m_{ad} = 3/2$, we get the solution where $\alpha = 4$.

\section{Density solution}
It's pretty straight-forward to find the solution for density now that we have the solution
for temperature. Returning to Eqn.~\ref{eqn:integrand}, we substitute $T = \rho^{1/m_{ad}}$ and
$dT = (1/m_{ad})\rho^{m_{ad}^{-1} - 1} d\rho$ to get
\begin{equation}
\frac{1}{m_{ad}}\rho^{-3/2 + m_{ad}^{-1}}d\rho = C\left(\beta \Gamma\tau^{-1/2}d\tau + (M_0 - \beta I_0)\tau^{-3/2}d\tau\right).
\end{equation}
In the limiting case where $m_{ad} = 2$, the LHS is $\rho^{-1}$ and is a log integral. For
cases where $m_{ad} < 2$, We integrate to get
\begin{equation}
(m_{ad}^{-1} - 0.5)^{-1}m_{ad}\rho^{m_{ad}^{-1} - 1/2}\bigg|_{\rho_0}^{\rho} = 2 C \xi(\tau)\bigg|_{\tau_0}^{\tau}
\end{equation}
I'll now define $\chi^{-1} = m_{ad}^{-1} - 0.5$, rearrange, and find
\begin{equation}
\rho = \left[\frac{2 C}{m_{ad}\chi}(\xi[\tau] - \xi[\tau_0]) + \rho_0^{1/\chi}\right]^\chi,
\end{equation}
and $\chi = 6$ for our $m_{ad} = 3/2$ solutions.

\section{Leading order evolution}
What we eventually find from all of our derivations here is that, to leading order in $\tau$,
\begin{gather}
\rho = \left(\frac{2 C\beta\Gamma}{m_{ad}\chi}\right)^{\chi} \tau^{\chi/2} \\
T    = \left(\frac{2 C \beta \Gamma}{\alpha}\right)^{\alpha} \tau^{\alpha/2} \\
w    = \frac{C B \beta}{(\grad T)_{ad}}\left(\frac{2 C}{\alpha}\right)^{\alpha - 1} \tau^{\alpha/2 - 1},
\end{gather}
where $\alpha = 2$ and $\chi = 3$ for $m_{ad} = 3/2$.

%\section{Entrainment rate}
%As in \cite{lecoanet&jeevanjee2018}, we want to know what the entrainment rate is, or
%\begin{equation}
%\epsilon = \frac{d \ln V}{dz} \sim 3 \frac{d\ln r}{dz}.
%\end{equation}
%We also know that
%\begin{equation}
%r = \sqrt{2 \tau / \rho} \sim \sqrt{2\left(\frac{m_{ad}\chi}{2 C \beta \Gamma}\right)^{\chi} \tau^{1-\chi/2}},
%\end{equation}
%so
%\begin{equation}
%\epsilon \sim \frac{3}{2}(1 - \chi/2)\frac{d\ln \tau}{dz} 
%= \frac{3}{2}(1 - \chi / 2) \frac{d \ln \tau}{d\tau}\frac{d\tau}{dt}\frac{dt}{dz}
%= \frac{3}{2}\frac{B}{\Gamma}(1 - \chi/2)\frac{1}{\tau w}
%\end{equation}
%For $\chi = 6$, $\alpha = 4$, this becomes
%\begin{equation}
%\epsilon \propto \tau^{-2}
%\end{equation}




\bibliography{biblio.bib}
\end{document}
