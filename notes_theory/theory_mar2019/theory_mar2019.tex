\documentclass[onecolumn, amsmath, amsfonts, amssymb]{aastex62}
\usepackage{mathtools}
\usepackage{natbib}
\usepackage{bm}
\newcommand{\vdag}{(v)^\dagger}
\newcommand\aastex{AAS\TeX}
\newcommand\latex{La\TeX}


\newcommand{\Div}[1]{\ensuremath{\nabla\cdot\left( #1\right)}}
\newcommand{\DivU}{\ensuremath{\nabla\cdot\bm{u}}}
\newcommand{\angles}[1]{\ensuremath{\left\langle #1 \right\rangle}}
\newcommand{\KS}[1]{\ensuremath{\text{KS}(#1)}}
\newcommand{\KSstat}[1]{\ensuremath{\overline{\text{KS}(#1)}}}
\newcommand{\grad}{\ensuremath{\nabla}}
\newcommand{\RB}{Rayleigh-B\'{e}nard }
\newcommand{\stressT}{\ensuremath{\bm{\bar{\bar{\Pi}}}}}
\newcommand{\lilstressT}{\ensuremath{\bm{\bar{\bar{\sigma}}}}}
\newcommand{\nrho}{\ensuremath{n_{\rho}}}
\newcommand{\approptoinn}[2]{\mathrel{\vcenter{
	\offinterlineskip\halign{\hfil$##$\cr
	#1\propto\cr\noalign{\kern2pt}#1\sim\cr\noalign{\kern-2pt}}}}}

\newcommand{\appropto}{\mathpalette\approptoinn\relax}
\newcommand{\pro}{\ensuremath{\text{Ro}_{\text{p}}}}
\newcommand{\con}{\ensuremath{\text{Ro}_{\text{c}}}}

\usepackage{color}
\newcommand{\gv}[1]{{\color{blue} #1}}

%% Tells LaTeX to search for image files in the 
%% current directory as well as in the figures/ folder.
\graphicspath{{./}{figs/}}

\begin{document}
\section{What we know}
From the simulations we've done so far, when viscous heating is neglected, we have the
following understanding of thermals:
\begin{enumerate}
\item The \emph{total entropy} of the thermal, $\rho s_1$, is conserved in the absence of
viscous heating.
\item For a low-$\epsilon$, low-Mach-number thermal, in which the total entropy is conserved
and the anelastic approximation is appropriate, we have two more pieces of knowledge:
\begin{enumerate}
\item From simple momentum equation arguments, the thermal's momentum increases linearly due
to the constant buoyant force, $\int \rho w dV \sim \rho w V = \rho s_1 g / c_p t = B t$.
\item From an anelastic-like impulse formulation with $\rho_1/\rho_0 \ll 1$, we also
know that the impulse linearly increases, $\rho \pi R^2 \Gamma = Bt$
\citep[][eqn 36 modified to buoyant vortex ring]{shivamoggi2010}.
\end{enumerate}
\item The thermal can be approximated as an ellipsoid whose principal semi-axes have lengths
$a = b = R$ and $c = h/2$, where $R$ is the radius of the thermal and $h/2$ is half the height
of the thermal volume. In other words, its horizontal cross-sections are circles and it's a bit
squished, so it's not quite a sphere. We can probably assume that $h \propto R$, and so
the volume of the thermal can be written $V = m R^3$ for some constant $m$.
\item The depth of the thermal follows a power law in time, $d = d_0t^\alpha$.
\item The circulation in the thermal, $\Gamma$, is constant.
\end{enumerate}

\section{Putting the pieces together}
From piece of knowledge 2 above, we see that $\rho V w \sim \rho R^2 \Gamma$. Plugging in
piece of knowledge 3, we find $R w \sim \Gamma / m$. Since $w$, the thermal downward velocity,
is the derivative of its depth, $w = \partial_t (d_0 t^\alpha) = \alpha d_0 t^{\alpha-1}$, 
we see that
\begin{equation}
R \sim \frac{\Gamma}{\alpha d_0 m} t^{1 - \alpha}.
\label{eqn:R_scale}
\end{equation}

With that figured out, we can go back to piece of knowledge numbers 2 and 3 above, and see
that $\rho V w \sim m \rho R^3 w \sim B t$, or that $(\Gamma/[\alpha d_0])^3 m^{-2} t^{3(1-\alpha) - (1-\alpha)}\rho = Bt$.
Dividing both sides by $t$ and rearranging a bit, we find
$$
\rho t^{1 - 2\alpha} = m^2 B \left(\frac{\alpha d_0}{\Gamma}\right)^3 = \text{const.}
$$
Taking a time-derivative, and rearranging, we find
\begin{equation}
\frac{\partial \ln \rho}{\partial t} = \frac{2\alpha - 1}{t}.
\end{equation}
So a simple $1 / t$ fit to $\partial \ln\rho / \partial t$, or perhaps even better, a moving
average of $t \partial_t \ln\rho$ (not what I'm doing right now), gives us the value of $\alpha$! 





\bibliography{biblio.bib}
\end{document}
