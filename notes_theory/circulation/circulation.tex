\documentclass[onecolumn, amsmath, amsfonts, amssymb]{aastex62}
\usepackage{mathtools}
\usepackage{natbib}
\usepackage{bm}
\newcommand{\vdag}{(v)^\dagger}
\newcommand\aastex{AAS\TeX}
\newcommand\latex{La\TeX}


\newcommand{\Div}[1]{\ensuremath{\nabla\cdot\left( #1\right)}}
\newcommand{\DivU}{\ensuremath{\nabla\cdot\bm{u}}}
\newcommand{\angles}[1]{\ensuremath{\left\langle #1 \right\rangle}}
\newcommand{\KS}[1]{\ensuremath{\text{KS}(#1)}}
\newcommand{\KSstat}[1]{\ensuremath{\overline{\text{KS}(#1)}}}
\newcommand{\grad}{\ensuremath{\nabla}}
\newcommand{\RB}{Rayleigh-B\'{e}nard }
\newcommand{\stressT}{\ensuremath{\bm{\bar{\bar{\Pi}}}}}
\newcommand{\lilstressT}{\ensuremath{\bm{\bar{\bar{\sigma}}}}}
\newcommand{\nrho}{\ensuremath{n_{\rho}}}
\newcommand{\approptoinn}[2]{\mathrel{\vcenter{
	\offinterlineskip\halign{\hfil$##$\cr
	#1\propto\cr\noalign{\kern2pt}#1\sim\cr\noalign{\kern-2pt}}}}}

\newcommand{\appropto}{\mathpalette\approptoinn\relax}
\newcommand{\pro}{\ensuremath{\text{Ro}_{\text{p}}}}
\newcommand{\con}{\ensuremath{\text{Ro}_{\text{c}}}}

\usepackage{color}
\newcommand{\gv}[1]{{\color{blue} #1}}

%% Tells LaTeX to search for image files in the 
%% current directory as well as in the figures/ folder.
\graphicspath{{./}{figs/}}

\begin{document}
\title{Circulation in an anelastic system}
\section{Momentum and Vorticity equations}
The anelastic momentum equation takes the form of eqn 27 in \cite{lecoanet&all2014},
\begin{equation}
\frac{\partial \bm{u}}{\partial t} + \bm{u}\cdot\grad\bm{u}
= -\grad \varpi + \frac{g}{c_P} S_1 \hat{z} + \Div{\stressT}.
\label{eqn:momentum}
\end{equation}
Taking the curl of Eqn.~\ref{eqn:momentum}, we can retrieve the vorticity equation in two
forms:
\begin{gather}
\frac{\partial \bm{\omega}}{\partial t}
= \nabla \times \left[ \bm{u}\times\bm{\omega} + \frac{g}{c_P} S_1 \hat{z} + \Div{\stressT}  \right]
\label{eqn:vorticity_1}
\\
\frac{\partial \bm{\omega}}{\partial t} + \bm{u}\cdot\grad\bm{\omega}
= -\bm{\omega}\DivU + \bm{\omega}\cdot\grad\bm{u} 
+ \nabla \times \left[ \frac{g}{c_P} S_1 \hat{z} + \Div{\stressT}  \right]
\label{eqn:vorticity_2}
\end{gather}

\section{Integral identities}
As circulation is defined as the path-integral of velocity (or the surface integral of
vorticity), it is useful to know a vector identity for each of these types of integrals.
For an arbitrary vector field, $\bm{Q}$, the langrangian derivative
($D/Dt = \partial/\partial t + \bm{u}\cdot\grad$) of its line integral around contour $C$ is
\begin{equation}
\frac{D}{Dt}\oint_C \bm{Q}\cdot d\bm{x} 
= \oint_C\frac{D \bm{Q}}{D t}\cdot d\bm{x} + \oint_C\bm{Q} \cdot d\bm{u}.
\label{eqn:line_integral}
\end{equation}
The lagrangian derivative of a surface integral along a surface $A$ is
\citep[Eqn.~4.45 of][]{choudhuri1998},
\begin{equation}
\frac{D}{Dt}\iint_A d\bm{S}\cdot\bm{Q}
= \iint_A d\bm{S}\cdot\left[\frac{\partial \bm{Q}}{\partial t} - \grad\times(\bm{u}\times\bm{Q})\right].
\label{eqn:area_integral}
\end{equation}

\section{Circulation from the momentum equation}
Starting with Eqn.~\ref{eqn:momentum} and using Eqn.~\ref{eqn:line_integral} with $\bm{Q} = \bm{u}$,
we acknowledge that the second term in Eqn.~\ref{eqn:line_integral} is
\begin{equation}
\oint_C \bm{u}\cdot d\bm{u} = \frac{1}{2} \oint_C d|\bm{u}^2| = 0,
\end{equation}
as it is a closed loop integral of a perfect differential. Thus, we have
\begin{equation}
\frac{D \Gamma}{D t} = \frac{D}{Dt} \oint_C \bm{u}\cdot d\bm{x} = \oint_C \frac{D\bm{u}}{Dt}\cdot d\bm{x}
= \oint_C \left[-\grad\varpi + \frac{g}{c_P} S_1 \hat{z} + \Div{\stressT}  \right]\cdot d\bm{x}
\end{equation}
Since the closed loop integral conservative force (e.g., $-\grad\varpi$) is zero, we are left
with
\begin{equation}
\frac{D \Gamma}{D t} 
= \oint_C \left[\frac{g}{c_P} S_1 \hat{z} + \Div{\stressT}  \right]\cdot d\bm{x}
\label{eqn:circulation_momentum}
\end{equation}

\section{Circulation from the vorticity equation}
Starting with Eqn.~\ref{eqn:vorticity_1} and using Eqn.~\ref{eqn:area_integral} with
$\bm{Q} = \bm{\omega}$, we rather simply retrieve
\begin{equation}
\frac{D \Gamma}{Dt} = \frac{D}{Dt}\iint_A d\bm{S}\cdot\bm{\omega}
= \iint_A d\bm{S} \cdot\left[\frac{\partial \bm{\omega}}{\partial t} - \grad\times(\bm{u}\times\bm{\omega})\right]
= \iint_A d\bm{S} \cdot\grad\times\left[\bm{u}\times\bm{\omega} + \frac{g}{c_P}S_1 \hat{z} + \Div{\stressT} - \bm{u}\times\bm{\omega}\right].
\end{equation}
Or, simplifying,
\begin{equation}
\frac{D \Gamma}{Dt} 
= \iint_A d\bm{S} \cdot\grad\times\left[\frac{g}{c_P}S_1 \hat{z} + \Div{\stressT}\right].
\label{eqn:circulation_vorticity}
\end{equation}
Assuming that Stokes' theorem holds for the contour $C$ around area $A$, the definitions
of circulation for Eqns.~\ref{eqn:circulation_momentum} and \ref{eqn:circulation_vorticity}
are equivalent, and only buoyant and viscous terms should have any effect on the circulation
in the thermal.


\bibliography{biblio.bib}
\end{document}
